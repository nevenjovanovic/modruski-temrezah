%-*-coding:utf-8;-*-
%&LaTeX
\documentclass[a5paper,twoside]{article}
\usepackage{polyglossia,fontspec}
\setmainlanguage{latin}
\setotherlanguage{croatian}

\defaultfontfeatures{Ligatures=TeX}

\usepackage{indentfirst,titling,enumitem}

% prilagodba dimenzija stranice
\usepackage[a5paper,pdftex]{geometry}%a4paper,pdftex
%\usepackage[a4paper,pdftex]{geometry}
\geometry{scale=1,paperwidth=161mm,paperheight=230mm,
bindingoffset=0in,top=27mm,inner=21mm,outer=26mm,width=117mm,height=180mm}%
\parindent=7mm

\usepackage[small,center,rm]{titlesec}
\titlespacing{\section}{0pt}{11mm}{7mm}

% titling
\pretitle{\begin{center}\vskip 5mm}
\posttitle{\par\end{center}}
\preauthor{}
\postauthor{}
\predate{}
\postdate{}

% slova
\setmainfont{Old Standard TT}
\setsansfont{Old Standard TT}

% makro za regum gesta
\newcommand{\kratica}{\emph{Regum gesta}}%

% reference bold
\newcommand{\refb}[1]{\textbf{\ref{#1}}}

\renewcommand{\labelenumi}{\Alph{enumi}}

%tekuće glave
%tekuće glave
\usepackage{fancyhdr}
\pagestyle{fancy}
\fancyhead{} % clearall
\fancyhead[RO,LE]{\thepage}
\fancyhead[EC]{\MakeUppercase{Nikola Modruški: Djela u službi Pape Siksta IV.}}
\fancyhead[OC]{\MakeUppercase{Fontes lectionvm}}
\fancyfoot{}
\renewcommand{\headrulewidth}{0.4pt}

\makeatletter
\renewcommand{\@makefnmark}{\mbox{%
\textsuperscript{\normalfont\@thefnmark\ }}}
\makeatother

\renewcommand\footnotesize{\fontsize{10}{10.5} \selectfont}
\renewcommand\tiny{\fontsize{9}{9.5} \selectfont}
\renewcommand\large{\fontsize{11}{11.5} \selectfont}
\renewcommand\Large{\fontsize{12.5}{13} \selectfont}

% fusnote i crta
\renewcommand{\footnoterule}{\vspace*{-7pt}
    \noindent\rule{2in}{0.4pt}\vspace*{6.6pt}}

\setlength{\headheight}{12.6pt}

% makro za naziv programa
\newcommand{\makeup}[1]{\MakeUppercase{\emph{#1}}}


% prilagodba quote
% prilagođavamo citat propozicijama
\renewenvironment{quote}
               {\list{}{\rightmargin 0mm
                \leftmargin 7mm
                \itemindent 0em}%
                \item\relax}
               {\endlist}

\newenvironment{opisi}{%
\let\olditem\item% 
\renewcommand\item[2][]{\olditem ##1 ##2 \itemsep 3pt}%
\begin{description}}{\end{description}%
}


%siročići i udovice
% \widowpenalty=10000
% \clubpenalty=10000

\begin{document}
%%tth:\begin{html}<META HTTP-EQUIV="CONTENT-TYPE" CONTENT="text/html; charset=utf-8">\end{html}

\hyphenation{his-to-ri-o-graf-ske po-li-tič-ko po-li-tič-ko-his-to-ri-o-graf-ske}
\hyphenation{quae-sti-o-nes quin-qua-gin-ta quo-que DXXXVIII Zuo-ni-me-ro}
\hyphenation{quad-ra-ta quo-mo-do Schwand-tne-ro-vi cro-a-tiae I-su-krs-ta}
\hyphenation{Pas-qua-li-go XVIII U-di-sla-va in-tel-le-xis-set kra-ljev-stvo}
\hyphenation{in-ua-se-runt quod pes-si-mo pe-dan-tno-an-ti-kvar-ski pas-cha-les}
\hyphenation{Pa-lu-šom MDCCXLVIII}

% No extra spacing after periods
\frenchspacing

% Font sizing
\fontsize{11}{13.2}
\selectfont

% veća čitljivost
\linespread{1.1}

%uvlaka paragrafa
\setlength{\parindent}{7mm}

% počinje na s. 5
\setcounter{page}{169}

\title{\Large{\MakeUppercase{Fontes lectionvm}}}

\date{}%Datum ove verzije: \today}

\maketitle 

%\texttt{Završna verzija, uredniku na čitanje.}
\thispagestyle{empty}
% [ova stranica namjerno prazna]


\section*{1.\thinspace LIBRI IMPRESSI}
\begin{description}[nolistsep,itemsep=3pt,font=\rmfamily]
\item[V] -- Romae, in domo Antonii et Raphaelis de Vulterris (anno 1474).
\item[Ge] -- Romae, typis excussit Johannes Gensberg (anno 1474).
\item[R] -- Rostochii, excusserunt fratres Domus Horti Viridis ad S. Michaelem, anno 1476.
\item[C] -- Paduae, excussit Matthaeus Cerdonis, 30 Aug. 1482.
\item[P] -- Romae, excussit S. Plannck, 1482.
\item[Gd] -- Romae, excussit Bartholomaeus Guldinbeck (qui ibi officinam habuit librariam annis 1484–1488).
\end{description}

\section*{2.\thinspace CODICES}

\bigskip

\begin{description}[noitemsep,itemsep=3pt,labelsep=5pt,font=\rmfamily]
\item[va] -- Codex Vaticanus: Bibliotheca apostolica Vaticana \emph{Vat.~lat.\ 8750}.
\item[m] -- Codex Monacensis: Bibliotheca Bavarica \emph{CLM 461}.
\item[pa] -- Codex Panormitanus: Panormi, Biblioteca centrale della Regione siciliana, \emph{I.B.6,} ff. 32r-54r.
\item[o] -- Codex Olomoucensis, sub fine saec. XV; Olomoucii, Vědecká knihovna; Textus varii; Historia Bohemica \emph{sign. M I 159}, ff. 170r–174v.
\item[ve] -- Codex Venetus, \emph{Marc.~Lat.\ cl.~XIII, 180 (4667).}
\item[co] -- Codex Corsinianus: cart. misc., saeculi XV; Romae, Accademia Nazionale dei Lincei, Biblioteca dell'Accademia dei Lincei e Corsiniana, fondo principale, \emph{Corsin. 583 (45 C 18),} ff. 123v-125v.
\end{description}


\clearpage
\thispagestyle{empty}
\hfill
\clearpage


\end{document}
