\documentclass[a5paper,twoside]{article}
\usepackage{polyglossia,fontspec}
\setmainlanguage{latin}
\setotherlanguage{croatian}

\defaultfontfeatures{Ligatures=TeX}

\usepackage[a5paper,pdftex]{geometry}
\geometry{scale=1,paperwidth=161mm,paperheight=230mm,
bindingoffset=0in,top=27mm,inner=21mm,outer=26mm,width=117mm,height=180mm}%
\parindent=7mm

\usepackage[series={A,B},noeledsec,nofamiliar,noledgroup]{reledmac}
\Xarrangement[A]{paragraph}
\Xnotenumfont[A]{\bfseries}
\Xparindent[A]
%\Xafterrule[A]{5pt}

\Xarrangement[B]{paragraph}
\Xnotenumfont[B]{\bfseries}
\Xendlemmafont{\bfseries}
\Xendbeforepagenumber{p. }
\Xendafterpagenumber{, }
\Xendlineprefixsingle{l. }
\Xparindent[B]
%\Xmaxhnotes[⟨s⟩]{⟨l⟩}
%\AtBeginDocument{
%\Xmaxhnotes[A]{0.2\textheight}
%\Xmaxhnotes[A,B]{0.75\textheight}
%}
%\Xbeforenotes[A,B]{2em plus 1em minus 1em}
\Xbeforenotes[B]{2.2em plus 1em minus 1em}
\Xbeforenotes[A]{2.2em plus 1em minus 1em}
\Xafterrule[A,B]{1em}
\Xnolemmaseparator[B]
\Xinplaceoflemmaseparator[B]{1em}

\Xendinplaceoflemmaseparator[B]{1em}
%\Xendlemmaseparator

\fnpos{%
{B}{critical},
{A}{critical}%
}

% sidenotes za brojeve paragrafa
\sidenotemargin{left}
\renewcommand{\ledlsnotefontsetup}{\normalsize}% left
\renewcommand{\ledrsnotefontsetup}{\normalsize}% right
\renewcommand{\ledlsnotewidth}{0.5\marginparwidth}
\renewcommand{\ledlsnotesep}{0.25pt}
\leftnoteupfalse
\rightnoteupfalse

%\setsidenotesep{ $|$ }

\setmainfont{Times New Roman}
%\setsansfont{Old Standard TT}
%veličina fusnota, marginalija, glavnog teksta
\renewcommand\footnotesize{\fontsize{10}{10.5} \selectfont}
\renewcommand\tiny{\fontsize{9}{9.5} \selectfont}
\renewcommand\large{\fontsize{11}{11.5} \selectfont}

% fusnote i crta
%\setlength{\skip\footins}{1.5\baselineskip}% plus0.5\baselineskip minus0.5\baselineskip}
%\setlength{\footnotesep}{1.2\baselineskip}%{12pt plus2pt minus2pt}
%\renewcommand{\footnoterule}{\vspace*{-7pt}
%    \noindent\rule{2in}{0.4pt}\vspace*{6.6pt}}

%tekuće glave
\usepackage{fancyhdr}
\pagestyle{fancy}
\lhead[\thepage]{}
\chead[\MakeUppercase{Nikola Modruški: Djela u službi Pape Siksta IV.}]{\MakeUppercase{Oratio in fvnere Petri Cardinalis S. Sixti}}
%]{}%\small\MakeUppercase{Djela dalmatinskih i hrvatskih kraljeva}}
%\chead[\small{Colloquia Maruliana XVIII (2009.)}]{\small{Regum Delmatię atque Croatię gesta}}
\rhead[]{\thepage}
\lfoot{}
\cfoot{}
\rfoot{}
\renewcommand{\headrulewidth}{0.4pt}



\begin{document}
% No extra spacing after periods
\frenchspacing

% Font sizing
\fontsize{11}{13.2}
\selectfont

% veća čitljivost
\linespread{1.1}

% prva stranica bez zaglavlja
\thispagestyle{empty}

%broji retke na svakoj stranici iznova
\lineation{page}
\linenummargin{outer}

\beginnumbering
\autopar


\pstart

\vspace*{2cm}

{\centering

\noindent ORATIO IN FVNERE REVERENDISSIMI DOMINI \\
DOMINI PETRI CARDINALIS SANCTI SIXTI \\
\edtext{HABITA}{\Afootnote{Romę \textit{add. co}}} A REVERENDO PATRE \\
DOMINO NICOLAO 
EPISCOPO \\
\edtext{MODRVSIENSI}{\Afootnote{Modnisiensi \textit{ve;} Modrisiensi \textit{co;} 1475 \textit{add. Ge}}}

}

\pend

\bigskip

\vspace*{\fill}



%\numberpstarttrue

%\setcounter{pstart}{1}

Cum\ledsidenote{1} in \edtext{omni}{\Afootnote{\textit{om. R ve pa co}}} funebri celebratione duo praecipue dicendi genera a maioribus nostris usurpari soleant, amplissimi patres, alterum quo tristes amicorum animos maerore leuarent, quibus cum \edtext{amici}{\Afootnote{amicis \textit{ve}}} uita suauissima extitit, mors ipsa iocunda esse non potuit, alterum quo 
\edtext{extremum amici munus}{\Bendnote{\textit{Verg. E. 8, 29:} extremum hoc munus morientis habeto; \textit{Verg. A. 4, 416:} Extremum hoc miserae det munus amanti; \textit{Lucan. 6, 624:} Ah miser, extremum cui mortis munus inique / Eripitur; \textit{Iuvenc. Evang. 4, 264:} Iam decedenti uesper succedere soli / Coeperat, et procerum solus cum iustior audet / Corpus ad extremum munus deposcere Christi; \textit{Iordanes, Getica 53:} et mox Vidimer Italiae terras intravit, extremum fati munus reddens rebus excessit humanis; \textit{Prudentius Trecensis (m. 861), Sermo de vita et morte Maurae, 115, 1374D:} Hoc extremum munus a te peto, Prudenti Pater episcope, ut in eorum praesentia de manu tua eucharistiae et unctionis extremae recipiam sacramenta}} 
rebus ab eo bene gestis uirtutumque ipsius copia ac splendore amplissimis laudibus \edtext{exornarent}{\Afootnote{exornaret \textit{Ge o}}} – 
\edtext{illud ego prius consolationis genus ita prorsus in omne tempus perdidi ut magis ipse solatio egeam}{\lemma{illud\dots\ solatio egeam}\Bfootnote{\textit{Cic.~fam.\ 5, 16, 1 (quam sententiam citauerat Nicolaus ipse alibi, v.\ De consolatione 1, 9, 10):} Etsi unus ex omnibus minime sum ad te consolandum accommodatus, quod tantum ex tuis molestiis cepi doloris ut consolatione ipse egerem}} 
quam ut illud cuiquam uel 
\edtext{praestare possim uel polliceri}{\lemma{praestare\dots\ polliceri}\Bendnote{\textit{Nepos Att. 15, 1, 3:} leves arbitrabatur polliceri quod praestare non possent; \textit{Suet. 3, 24, 1, 2:} alter coram exprobraret ceteros, quod polliciti sint tarde praestare, sed ipsum, quod praestet tarde polliceri; \textit{Leonardus Brunus Aretinus (1369-1444), Historiarum Florentini populi Libri XII, 3, 176:} Tu dudum in purgando tutandoque regno tuo occupatus, hoc illis quod eorum fides et genus tuum polliceri videbatur, praestare supersedisti; \textit{Giannantonio Campano (1429-1477), Carminum appendix, 1, 8, 8:} Disce polliceri / Quae praestare potes}}. \edtext{Quod}{\Afootnote{Quid \textit{R}}} etiam si 
\edtext{minime perdidissem}{\Bendnote{\textit{Cassiod. Expositio in Psalterium, 70, 0373C:} spem conversionis minime perdiderunt; \textit{Liutprandus Cremonensis (922–972), Antapodosis, 25, 189:} omnium iudicio coronam recepit et caligarum decorem minime perdidit; \textit{Petrus Abaelardus (1079–1142), Epistolae, 178, 0373B:} passiones, quibus frequenter quatiuntur, qui tale aliquid minime perdiderunt; \textit{Petrus Cluniacensis (1092–1156), Epistolae, 189, 0074C:} qui talibus unguentis Dominum inunxerunt, fructum devotionis suae apud eum minime perdiderunt}}, \edtext{numquam tamen \edtext{dispicere}{\Afootnote{despicere \textit{Gd}}} possem}{\lemma{numquam\ldots\ possem}\Afootnote{\textit{add. postea scriba in pa}}} qua oratione aut quibus rationibus etiam illi quidem, qui 
\edtext{principes eloquentiae}{\Bendnote{\textit{Cic. de or. 1, 1, 1, 13; 19:} Atque ut omittam Graeciam, quae semper eloquentiae princeps esse voluit; \textit{1, 3, 19, 63; 94:} sententiae atque eloquentiae principem in senatu, in populo, in causis publicis; \textit{Aug. doctr. Christ. 34:} Nam et ipsos Romanae principes eloquentiae non piguit dicere; \textit{Petrarca, De remediis utriusque fortune, 1, 9; 35:} neque dubium sit quod eloquentie princeps in Rethoricis scribit}} sunt habiti, hanc tam grauem sacrosancti senatus uestri iacturam et hunc publicum totius curiae luctum ulla ex parte leuare possent. Perdidistis enim, patres amplissimi, praestantissimum collegam uestrum, cuius suauissimam consuetudinem, 
\edtext{comitatem, benignitatem, liberalitatemque}{\lemma{comitatem\dots\ liberalitatemque}\Bendnote{\textit{Cic. de fin. 5, 23, 65; 4:} iustitia dicitur, cui sunt adiunctae pietas, bonitas, liberalitas, benignitas, comitas, quaeque sunt generis eiusdem}} quotidie experiebamini, cuius \edtext{ingenii dexteritatem}{\Bendnote{\textit{Liv. 28, 18, 4:} ad omnia naturalis ingenii dexteritas; \textit{Pontano, De amore coniugali, 3, 3, 34:} qui reddat utrunque / Scitus avum ingenii dexteritate, puer}} et incredibilem consilii prudentiam \edtext{indies}{\Afootnote{in dies \textit{R Gd P m va pa}}} magis admirabamini. Amisistis summi pontificis, patris uestri piissimi, \edtext{singulare solacium et sacrae eius senectutis optatissimum baculum}{\lemma{singulare solacium\dots\ baculum}\Bfootnote{\textit{Tob, 10, 4:} baculum senectutis nostrae, solatium vitae nostrae}\Bendnote{\textit{Aug. ep. 185-270 [CSEL], 266, 3, 3:} laborum periculorumque nostrorum singulare solacium est; \textit{Alcuinus (730–804), Epistolae, 100, 0145D:} Et hoc mihi singulare solatium in spiritu consolationis sanctae Dominus Christus perdonare dignatus est; \textit{Thomas Aquinas, Officium corporis Christi Sacerdos, 5, 185:} de sua contristatis absentia solatium singulare; \textit{Thomas a Kempis (1380–1471), De imitatione Christi, 4, II, 23:} Laetare, anima mea, et gratias age Deo pro tam nobili munere et solatio singulari in hac lacrymarum valle tibi relicto}}, \edtext{participem secretorum}{\Bendnote{\textit{Tac. A. 1, 6, 6:} quod postquam Sallustius Crispus particeps secretorum (is ad tribunum miserat codicillos) comperit; \textit{Hilarius Pictaviensis, De Trinitate, 10, 0266C:} O dignus ille magnorum et coelestium arcanorum conscientia, et assumptus secretorum divinorum electusque particeps; \textit{Hieronymus, ep. 22, 0386:} sicuti Ioannes super pectus Iesu recubuit, qui secretorum eius effectus est particeps; \textit{Petrus Damianus (1007–1072), Sermones, 144, 0738C:} Inest et plerisque bonis illuminatione, quos de futurorum cognitione nobilitat et participes suorum efficit secretorum}}, laborum socium, peregrinationis comitem, 
\edtext{leuamen curarum}{\Bendnote{\textit{Verg. Aen. 3, 709:} Heu, genitorem, omnis curae casusque leuamen; 
\textit{Sen. Med. 548:} Curis leuamen. spiritu citius queam;
\textit{Mart. epigr. 6, 68, 5:} Hic tibi curarum socius blandumque leuamen; 
\textit{Cassiodor. Variae, p1, 5, 104:} intrepidus quidem, sed reverenter astabat, opportune tacitus, necessarie copiosus, curarum nostrarum eximium levamen
}}, et per quem totius orbis principibus fidissima responsa et reddere consueuerat et accipere. \edtext{Absit uero inuidia}{\Bendnote{\textit{Liv. 9, 19, 15:} absit inuidia uerbo et ciuilia bella sileant}} et 
\edtext{liuor edax}
{\Bendnote{\textit{poetice; Ovid. am. 1, 15, 1:} Quid mihi, Liuor edax, ignauos obicis annos; \textit{rem. 389:} Rumpere, Liuor edax: magnum iam nomen habemus; \textit{Sen. Phaedr. 493; Lucan. Phars. 1, 288; Mart. epigr. 11, 33, 3; Paul. Nol. carm. 28, 287; Cypr. Gall. iud. 462; Paul. Petric. Mart. 2, 44, etc.}}}
 saltem parcat cineribus. \edtext{Extinctus}{\Afootnote{Extinctis \textit{Gd P}}} iacet optimarum artium deditissimus 
% \edtext{amator, interiit omnium \edtext{studiosorum}{\Afootnote{studiorum \textit{Ge va o}}} praecipuus fautor, cultor bonorum}{\lemma{amator\dots\ fautor, cultor bonorum}\Bendnote{\textit{Liv. 9, 46, 12:} integer populus, fautor et cultor bonorum; \textit{Alcuinus, Carmina, 101, 0840C:} Vir bonus et iustus, largus, pius, atque benignus; Catholicae fidei fautor, praeceptor, amator; Ecclesiae rector, doctor, defensor, alumnus, Iustitiae cultor, legis tuba, praeco salutis, Spes inopum, orphanisque pater, solator egentum}}
amator, interiit omnium \edtext{studiosorum}{\Afootnote{studiorum \textit{Ge va o}}} praecipuus \edtext{fautor, cultor bonorum}{\Bendnote{\textit{Liv. 9, 46, 12:} integer populus, fautor et cultor bonorum}}, curiae splendor, 
\edtext{ornamentum ciuitatis}{\Bendnote{\textit{Cic. Planc. 9, 8:} Flos enim equitum Romanorum, ornamentum civitatis, firmamentum rei publicae publicanorum ordine continetur; \textit{Aug. Sermones de Scripturis, 38, 0505:} Ipse homo ornamentum civitatis, ipse homo inhabitator, rector, gubernator civitatis; \textit{Leonardus Brunus Aretinus, Historiarum Florentini populi libri XII, 3; 631:} nobilitas ferme tota aberat, maximum profecto civitatis ornamentum}}, 
et huius \edtext{urbis}{\Afootnote{orbis \textit{va}}} diligentissimus restaurator. Corruit praeclarum magnanimitatis exemplar; cecidit 
%\edtext{Corruit praeclarum magnanimitatis exemplar; cecidit}{\lemma{Corruit\dots\ cecidit}\Bendnote{\textit{Hier. Translatio Homiliarum in Ieremiam et Ezechielem, 25, 0626C:} Cecidit quippe, et de sublimi corruit, venientibus nobis in locum istum miseriarum; \textit{Gregorius I, Dialogi, 77, 0260B:} ex eodem monte cecidit, et usque ad vallem corruit; \textit{Beda Incertus, Excerptiones, 94, 0546C:} Vidi bipedem super tripodem sedentem, cecidit bipes, corruit tripes; \textit{Honorius Augustodunensis (m. 1158), Expositio in Psalmos, 172, 0296A:} Per illam angelus de coelo cecidit; per istam homo de paradiso corruit; \textit{Aelredus Rievallensis (1110–1167), Sermones de oneribus, 195, 0439B:} Cecidit de coelo, et in terram corruit}} 
\edtext{munificentiae}{\Afootnote{manificentie \textit{pa}}}, \edtext{gratitudinis}{\Afootnote{gratitudis \textit{C}}}, et totius liberalitatis \edtext{alumnus}{\Afootnote{alumpnus \textit{V Gd P m}}}. Cuius iactura cum uniuersis lugenda sit, tum mihi praecipue atque \edtext{his}{\Afootnote{iis \textit{va}}} infelicissimis conseruis meis quibus haec crudelis et dira mors tam benignissimum abstulit dominum, interuertit benefactorem, ademit praesidium, et unico atque eodem piissimo patre nos acerba orbauit. \edtext{Nolite igitur, nolite}{\Afootnote{\textit{alterum} nolite \textit{om. Gd P m}}} expectare, praestantissimi patres, ut luctum \edtext{ac}{\Afootnote{et \textit{m}}} maerorem, in quo et \edtext{uos}{\Afootnote{nos \textit{va}}} nunc esse intueor et me ac totam hanc miserandam familiam, quoad uixerimus, fore necesse est, \edtext{uobis}{\Afootnote{nobis \textit{va}}} adimam. Quin potius pro uestra clementia dabitis dolori meo ueniam si eius acerbitate abductus nec rerum ordinem seruare ualeam nec uerborum tenere modum, praesertim cum ea culpa libentius ego carere uellem quam illam a uobis deprecari si mihi impositum onus detractare licuisset; agam tamen ut potero et minus temeritatis quam ingratitudinis notam subire uerebor.

Dicturus\ledsidenote{2} igitur de laudibus reuerendissimi domini domini Petri cardinalis Sancti Sixti, cuius miserandum funus hodierna die \edtext{celebratur}{\Afootnote{celebrantur \textit{va}}}, eas laudes, quas uel a parentibus uel a patria ipsius colligere poteram, hoc loco praetermittendas putaui; non quod illas aut obscuras aut tenues fore \edtext{duxerim}{\Afootnote{\textit{ita Ge C va o;} dixerim \textit{V R Gd P m pa ve co}}}, quoniam et honestissimis nobilissimisque ciuitatis suae parentibus est ortus et celeberrimo uetustoque \edtext{Ligurum}{\Afootnote{Ligurorum \textit{P}}} oppido \edtext{Saona}{\Afootnote{Soana \textit{P m}}}, uerum quod ipse illis tanto decori ac ornamento fuerit ut toto in orbe extremisque terrarum finibus amplissimis laudibus summaque gloria et celebrantur nunc et omnibus futuris seculis non desinent celebrari. Quae quidem tametsi satis grandis eius gloria sit, \edtext{qui}{\Afootnote{quis \textit{R}}} maioribus suis tam insignia uirtutis ornamenta dederit potius quam ab illis \edtext{acceperit}{\Afootnote{accepit \textit{va}}}, habeo tamen et alias immortales ac propemodum diuinas animi ipsius laudes (ut 
\edtext{fortunae corporisque quaelibet ingentia bona tamquam aliena}{\lemma{fortunae\dots\ aliena}\Bfootnote{\textit{cf.\ Cic.~Tusc.\ 5, 30, 85:} tria genera bonorum, maxuma animi, secunda corporis, externa tertia, ut Peripatetici nec multo veteres Academici secus; \textit{Laurentius Valla, De voluptate sive de vero bono 1, 16, 1:} ipsa testatur consensio communis que vulgato sermone appellat bona animi, bona corporis, bona fortune}} 
relinquam): pietatem, magnitudinem animi, munificentiam, prudentiam, modestiam, atque iustitiam; quae quales in eo fuerint, breuiter explicare conabor. \edtext{Qua}{\Afootnote{Qui \textit{ve}}} igitur pietate primum erga Deum fuerit quamque magnificus cultor ipsius, si aduixisset, futurus erat, in primo uitae suae limine clarissime demonstrauit. Annos natus duodecim cum orbatam patre familiam tanta prudentia regeret ut nec mater, matronarum praestantissima, fratres eius parentis sentirent desiderium, coepit Deo dicatum pectus zelo religionis feruescere. \edtext{Clarescebat}{\Afootnote{Clarescat \textit{Gd P m;} Clare sciebat \textit{C}}} autem iam tunc nomen \edtext{religiosissimi}{\Afootnote{religisissimi \textit{R}}} doctissimique uiri, \edtext{magistri Francisci}{\Afootnote{maijstri Francissi \textit{R}}}, conciuis et auunculi sui, nunc summi \edtext{pontificis}{\Afootnote{pontificatus \textit{va}}} papae Sixti, qui per id tempus Senis suis fratribus Sacras Scripturas interpretabatur. Hunc optimum Christianae militiae magistrum optimus futurus discipulus, quamuis puerili aetate, uirili tamen sensu sibi delegit; ad quem a religioso quodam sene multis exorato precibus, inscia matre, perductus est; diuino, ut \edtext{opinor}{\Afootnote{oppinor \textit{Ge}}}, nutu futurus ad apostolatum tam strenuus minister ad futurum sedis apostolicae mittitur antistitem. Quem ubi conspexisset Franciscus iam \edtext{religionis}{\Afootnote{religiosus \textit{R}}} \edtext{ueste}{\Afootnote{iuste \textit{Gd P}}} indutum, quam idcirco iuuenis in itinere assumpserat quo se facilius \edtext{magistro}{\Afootnote{maijstro \textit{R}}} suo insinuaret, multis eum hortatus est ut ad suos remearet et matris fratrumque curam, ut coeperat, ageret, uel maturiorem \edtext{domi}{\Afootnote{domini \textit{Ge C R va pa o ve co}}} praestolaretur aetatem, \edtext{quae}{\Afootnote{qua \textit{o}}} pati melius iugum Christi posset. Sed cum pueri constantiam nullis blanditiis, nullis persuasionibus, nullis denique minis euincere \edtext{posset}{\Afootnote{possent \textit{C}}}, diuinum, ut erat, in eo aliquod munus arbitratus, hortantibus fratribus, diui eum Francisci sacris initiauit, seruatisque pro more religionis rite caerimoniis uestem Christi induit. Qua assumpta ita omnia \edtext{tirocinii rudimenta}{\Bendnote{\textit{Iustin. epit. hist. Phil. 9, 1, 8:} tirocinii rudimenta deponeret}} libens promptusque et perdiscebat et exsequebatur ut nemo dubitaret et prudentiam illi et uires ante aetatem non nisi diuinitus subministrari. Quas ob res omnibus carus, omnibus dilectus esse coepit, praecipue autem ipsi \edtext{auunculo}{\Afootnote{auunclo \textit{pa}}} suo qui, diuina eius indole mirifice delectatus, piissimo sanctissimoque eum \edtext{amplectebatur}{\Afootnote{amplexabatur \textit{m}}} affectu. Vnde sibi curandum statuit ut tam excellens ingenium per bonas artes excoleretur. Itaque docto cuidam grammatico Latinis eum litteris Vicheriae imbuendum tradidit; quibus mira celeritate perceptis mox Ticinium, deinde Patauium, subinde Venetias, \edtext{Bononiam}{\Afootnote{Bononia \textit{m}}}, Perusium, Senam, \edtext{Ferrariamque}{\Afootnote{\textit{ita Gd C va m pa;} Ferariamque \textit{V Ge R ve o co}; Ferreriamque \textit{P}}} misit ut, quaecumque in tam celeberrimis Italiae \edtext{gymnasiis}{\Afootnote{gymnasijs \textit{ve;} ginnasiis \textit{V Ge R C P Gd va o pa;} gynasiis \textit{m}; gynnasijs \textit{co}}} aut liberalium artium aut sacrarum litterarum dogmata florerent, ipse uelut operosa apis undique colligeret et in sacram pectoris sui cellulam diligenter reconderet. Quod quidem intra breues annos adeo consecutus est ut credibile non sit uirum 
\edtext{ad agendum magis quam ad philosophandum}{\lemma{ad agendum\dots\ ad philosophandum}\Bfootnote{\textit{Cf.\ Platonis Symposium, tr.\ Ficinus (c.~1460)}: perinde ac tu nunc, qui quodlibet aliud agendum censes potius quam philosophandum}} natum \edtext{tantam}{\Afootnote{\textit{om. R;} tantum \textit{va}}} omnium scientiarum notitiam adeo breuiter percipere \edtext{potuisse}{\Afootnote{potuisset \textit{C pa}}}. Habeo hic testes complures, familiares eius, \edtext{uiros}{\Afootnote{ueros \textit{P m}}} quidem doctissimos, \edtext{quibuscum}{\Afootnote{quibus cum \textit{co}}} inter cenandum de uariis disciplinarum studiis frequenter disserere consueuerat adeo acute adeoque prompte ac subtiliter de quaestione \edtext{proposita}{\Afootnote{preposita \textit{pa}}} ut eum \edtext{putares}{\Afootnote{putaret \textit{Ge C o;} putare \textit{V Gd P m}}} \edtext{die}{\Afootnote{diu \textit{C}}} noctuque nulli \edtext{adeo}{\Afootnote{\textit{om. ve pa}}} alii rei quam euoluendis theologorum philosophorumque libris uacare. Tenebat \edtext{fixa memoriae}{\Bendnote{\textit{Cic. Cat. 4, 11, 1:}  nihil a vobis nisi huius temporis totiusque mei consulatus memoriam postulo: quae dum erit in vestris fixa mentibus, tutissimo me muro saeptum esse arbitrabor; \textit{Aug. de Trin. 42, 0999:} Possunt autem et pecora et sentire per corporis sensus extrinsecus corporalia, et ea memoriae fixa reminisci; \textit{Gregorius I, Moralia II, 76, 0150A:} perpetrata culpa in memoria fixa retinetur}} quaecumque ab ineunte aetate \edtext{a}{\Afootnote{\textit{om. R}}} praeceptoribus audierat. Vigebat praeterea stupendo ingenii acumine cuius perspicacitate facile in abditissima quaeque naturae secreta penetrabat et ex perceptis semel principiis difficillima quaeuis uel philosophiae \edtext{uel theologiae}{\Afootnote{\textit{om. V Gd P m}}} \edtext{problemata}{\Afootnote{probleumata \textit{o;} propleumata \textit{m}}} \edtext{summa}{\Afootnote{summam \textit{R;} summe \textit{o}}} cum omnium admiratione absoluebat; uersus complures multosque grammaticae textus, quos olim \edtext{puer}{\Afootnote{\textit{om. R}}} \edtext{edidicerat}{\Afootnote{edicerat \textit{P m}}}, ita memoriter recitabat ut ea \edtext{illum}{\Afootnote{eum \textit{o}}} heri aut \edtext{nudiustertius}{\Afootnote{nudius tertius \textit{co}}} memoriae mandasse existimares. 

Perfectis\ledsidenote{3} igitur \edtext{quam \edtext{celerrime}{\Afootnote{celeberrime \textit{m}}}}{\Afootnote{\textit{om. va}}} omnium bonarum artium studiis rediit sollicitus ad magistrum, munus, cui caelitus destinatus erat, optime impleturus.  Inerat enim menti eius praesaga quaedam futurorum diuinitas complurimaque, \edtext{antequam}{\Afootnote{ante quam \textit{co}}} \edtext{contigissent}{\Afootnote{contigisset \textit{ve}}}, 
%\edtext{per quietem discebat}{\Bendnote{\textit{Iustin. epit. hist. Phil. 12, 10; 3:} per quietem regi monstrata  in remedia veneni herba est \textit{(similia etiam apud Suet., Min. Fel., Solin. etc)}}} ita ut \edtext{insomnia}{\Afootnote{insonia \textit{Gd P}}}
per quietem discebat ipsius non uisa solum, sed certa uiderentur oracula. Quibus uarie sollicitatus coepit auunculum suum hortari, \edtext{coepit}{\Afootnote{cedit \textit{R}}} \edtext{importunius}{\Afootnote{importunis \textit{V P m;} importunus \textit{co}}} \edtext{compellere}{\Afootnote{complere \textit{Gd P m}}} Romam peteret atque in ea urbe uersari uellet in qua a maximo omnium \edtext{rectore Deo}{\Afootnote{Deo rectore \textit{va}}} futurus summus pontifex \edtext{erat designatus}{\Afootnote{designatus erat \textit{va}}}, ubi et \edtext{seipsum}{\Afootnote{se ipsum \textit{co}}} accepturum ab eo cardinalatus decus et alios uniuersos honores, quibus eum functum uidimus, mira asseueratione praedicebat. Cerno hic \edtext{nonnullos}{\Afootnote{non nullos \textit{co}}} praelatos et ex aliis ordinibus \edtext{uiros}{\Afootnote{unos \textit{Gd}}} \edtext{praestantes}{\Afootnote{praestante \textit{P m}}} a quibus magna cum \edtext{attestatione}{\Afootnote{actestatione \textit{pa}}} audiui singula haec, quae gessit, multis ante annis ab ipso praedicta fuisse.  Itaque repugnantem \edtext{magistrum}{\Afootnote{maijstrum \textit{R}}} et se tanto fastigio indignum reclamantem urgere non destitit donec multis et signis et prodigiis euictum perpulit Romam proficisci; in qua constitutus minime conquieuit \edtext{antequam}{\Afootnote{ante quam \textit{co}}} demandatum ab altissimi prouidentia munus uiriliter absolueret et \edtext{magistrum}{\Afootnote{maijstrum \textit{R}}} suum per uaria \edtext{bonorum}{\Afootnote{honorum \textit{va}}} incrementa ad summum \edtext{apostolatus culmen}{\Afootnote{culmen apostolatus \textit{va}}} conscendisse uideret; pro quibus laboribus et pro tam diligenti nauata opera eadem ipsa diuina prouidentia, quae \edtext{semper}{\Afootnote{\textit{om. R}}} \edtext{infirma mundi}{\Afootnote{\textit{om. va}}\Bfootnote{\textit{I Cor 1, 27:} infirma mundi elegit Deus ut confundat fortia}} eligere consueuit ut fortia quaeque confundat, \edtext{cardinalis}{\Afootnote{cardinalatus \textit{C}}} eum dignitatis splendore uoluit illustrare omnibusque mundi principibus ostendere quo ministro et ex quam humili loco \edtext{accepto}{\Afootnote{\textit{in margine addidit scriba in pa}}} uoluerit in sui uicarii \edtext{assumptione}{\Afootnote{assuptione \textit{pa}}} uti, ut discerent uniuersi \edtext{ueram}{\Afootnote{\textit{om. pa}}} certamque esse illam \edtext{Nabuchdenosoris}{\Afootnote{Nabuchdenosori \textit{Ge;} Nabuchodenosoris \textit{C;} Nabuchodenosori \textit{o;} Nabuchodonosor \textit{va}}} confessionem in quam et regno et sensui restitutus supplex prorupit dicens \edtext{in manu solius omnipotentis Dei esse omnia regna terrarum atque illa, quibus ipse \edtext{uoluerit}{\Afootnote{uoluerint \textit{P m}}} tradi, nec \edtext{esse}{\Afootnote{eā \textit{Ge o;} ea \textit{va}}} \edtext{alium}{\Afootnote{aliud \textit{o}}} quemquam \edtext{qui eius}{\Afootnote{eius qui \textit{o}}} possit obsistere \edtext{uoluntati}{\Afootnote{uolunptati \textit{va}}} aut dicere illi `quare sic fecisti'}{\lemma{in manu\dots\ sic fecisti}\Bfootnote{\textit{Cf.\ Daniel 3, 31–33}}}.

Petrus\ledsidenote{4} igitur per hunc modum \edtext{in principatu constitutus, qui solet uiros ostendere}{\lemma{in principatu\dots\ ostendere}\Bfootnote{\textit{Arist. Ethica ad Nicomachum 3, 1330, a 1-2:} principatus virum ostendit}}, talem \edtext{sese}{\Afootnote{se se \textit{co;} se \textit{ve}}} exhibuit qualem uix optare poteramus, non sperare. Assumpsit enim cum sublimi magistratu sublimes animos et spiritus tanti imperii \edtext{maiestate}{\Afootnote{maiestatate \textit{V;} et maiestate \textit{va}}} dignos \edtext{simulque}{\Afootnote{et simulque \textit{Ge C va o}}} cum eis \edtext{magnanimitatem}{\Afootnote{magnimitatem \textit{V}}}, clementiam, munificentiam, et \edtext{ceteras}{\Afootnote{cetaras \textit{Ge}}}, \edtext{quas prius}{\Afootnote{prius quas prius \textit{V P m}}} commemorauimus, \edtext{imperantium}{\Afootnote{\textit{om. ve;} in p-tium \textit{co;} in partium \textit{pa}}} uirtutes, quibus et ausus est cum regibus et \edtext{principibus}{\Afootnote{princibus \textit{C}}} magna gloria non segnius contendere quam si in illis natus educatusque fuisset, ut coepta \edtext{eius}{\Afootnote{\textit{om. R va}}} declarant aedificia, totque magnificentissimo cultu celebrata conuiuia, et \edtext{supellex}{\Afootnote{suppllex \textit{P}}} imperialibus \edtext{fastibus}{\Afootnote{fascibus \textit{Ge C va o m pa}}} digna. Turpe enim et \edtext{indecorum}{\Afootnote{inde eorum \textit{Gd}}} merito ducebat in hoc totius orbis capite, in hac prima Christianae religionis sede, ad quam \edtext{adorandam}{\Afootnote{adornandam \textit{m}}} imperatores, reges, et cuncti ferme principes terrarum uentitare solent, non talem esse supellectilem, non talia \edtext{extare palatia}{\Afootnote{ortare pallacis \textit{P}}}, quibus eos summus pontifex et suscipere honorifice posset et pro sua ipsorumque dignitate splendide honorare. Vnde et in hos usus omnia illa se \edtext{comparare}{\Afootnote{comperare \textit{Ge C Gd o}}} affirmabat, nec sibi, sed summis pontificibus, quicquid \edtext{praeparabat, \edtext{componere}{\Afootnote{componeret \textit{Ge o}}}}{\Afootnote{componeret, praeparabat \textit{va}}}. Affluebant quotidie opes et ab omnibus ferme Christianis principibus magni prouentus ultro offerebantur; quos licet ipse in illos, quos diximus, acceptaret usus, prophetici tamen documenti memor 
\edtext{minime ad ea cor apponebat}{\Bfootnote{\textit{Ps 61, 11} diuitiae si affluant, nolite cor apponere; \textit{quem locum citauerat Nicolaus etiam libro De consolatione 2, 15, 24}}}, neque \edtext{illis}{\Afootnote{ullo \textit{C}}} auaro \edtext{deuincebatur}{\Afootnote{euincebatur \textit{ve}}} affectu. Hinc est quod nec sciebat numerum nec omnino quid haberet aut quid impenderet nosse \edtext{uolebat}{\Afootnote{ualebat \textit{Gd P}}}; nullas a ministris impensarum exigebat rationes, nulla \edtext{computa}{\Afootnote{comput \textit{P}}} exigere uolebat. Iactura rerum ea dumtaxat mouebatur quae negligentia contigisset, culpam magis dolens quam \edtext{damnum}{\Afootnote{dampnum \textit{V C Gd m}}}. 

A\ledsidenote{5} suscepto semel negotio nullo metu absterreri potuerat, nullo periculo depelli. \edtext{Verax in sermone}{\Bendnote{\textit{Boeth. De disciplina scholarium, 64, 1235B:} in universa morum honestate oportet ut polleat praeclarius, ut sit utique in sermone verax, in iudicio iustus, in consilio providus, et in commisso fidelis, constans in vultu, pius in affatu, virtutibus insignitus, bonitateque laudabilis existat; \textit{Acta S. Sebastiani, 17, 1021C:} Erat enim vir totius prudentiae, in sermone verax, in iudicio iustus, in consilio providus, in commisso fidelis, in interventu strenuus, in bonitate conspicuus, in universa morum honestate praeclarus}}, in \edtext{facto}{\Afootnote{faco \textit{P}}} fidelis, in proposito constans, in peragendo strenuus, secretorum tenacissimus, et promissorum firmissimus \edtext{obseruator}{\Afootnote{osservator \textit{ve}}}. Sed illud mea sententia uincit uniuersa quod nullis mouebatur iniuriis, nullis offensis laedi posse uidebatur, nec \edtext{ullius}{\Afootnote{uillius \textit{P}}} rei facilius quam inimicitiarum obliuiscebatur. Floccipendebat aemulos quoscumque et nullam in rem pronior quam in supplicium ueniam uidebatur, tantumque ab omni ulciscendi ardore aberat ut inimicis suis benefacere gauderet. Nihil in se fictum, nihil subdolum aut simulatum esse uolebat, nec quicquam magis detestabatur quam mentitam probitatem. Mali quippe et iniqui hominis esse dicebat meliorem se \edtext{foris}{\Afootnote{fori \textit{Ge C;} fortis \textit{V P m}}} ostendere quam gerere domi, proborum autem uirorum esse non simulatione, sed integritate uiros superare. Sane munificentiae liberalitatisque eius largitatem quis est qui ignoret, nisi qui illa uti uel \edtext{noluerit}{\Afootnote{uoluerit \textit{o}}} uel nescierit? Ea ipse ita flagrabat ut illos etiam, quibus iustis de causis aliquando irascebatur, muneribus tamen ornare non cessaret. Non semel neque solus interfui cum quosdam familiarium merito obiurgasset, deinde uestimentis et magistratu donauit. Et cum aliquando a magistro domus suae, in cuius diligentissima fidissimaque cura merito conquiescebat, liberius argueretur quod nimia \edtext{indulgentia et}{\Afootnote{indulgenti et a \textit{P m}}} largitate domesticos faceret insolentiores, \edtext{placida}{\Afootnote{placenda \textit{P m}}} uoce respondit: ``Tuum est familiares meos pro neglecto officio \edtext{corrigere}{\Afootnote{corriger \textit{Gd}}}, meum autem \edtext{pro}{\Afootnote{\textit{om. P m}}} illorum in me amore congruis praemiis afficere; id faciendo uterque nostrum suo munere optime functus erit.'' O praeclaram uocem, o sententiam summo principe dignam; nec impunitatem erratorum laudauit, nec liberalitatem suam ullis male merentium factis occludi passus est. Felices quibus illa perfrui licuit, et nunc omnium infortunatissimos quibus tam crudeli fato \edtext{erepta}{\Afootnote{arrepta \textit{C}}} est! Quingentos ferme pascebat familiares, partim illustri, partim nobili, omnes honesto loco natos; praelatos, milites, doctores, oratores, poetas, aut alicui alii honestae arti deditos; nullis tantorum sumptibus, nullis grauabatur impensis. Hospitem enim \edtext{sese}{\Afootnote{se se \textit{co}}} \edtext{omnium}{\Afootnote{\textit{om. va}}} honestorum uirorum esse dicebat; quod ipsa 
\edtext{ueritate erat uerius}{\lemma{ueritate\dots\ uerius}\Bendnote{\textit{Cf. Sen. ep. 66, 8:} nihil invenies rectius recto, non magis quam verius vero, quam temperato temperatius; \textit{Sen. nat. 2, 34, 2:} In quo mihi falli uidentur. Quare? Quia uero uerius nihil est; \textit{Ficinus in versione Plotini (a. 1492 primum impressa), Plot. Enn. 5, 5:} Nihil enim potest ueritate uerius inueniri.}}. 
Nam si quis diligentius rem ipsam consideret, comperiet proculdubio omnes cardinalium domos nihil aliud esse quam 
\edtext{honesta hospitia}{\Bendnote{\textit{Coelestinus II (m. 1144), Epistola et privilegia, 179, 0794A:} Ideoque pro honestate et religione vestra dignum duximus ut in Romana urbe vobis ecclesiam concedamus, in qua et Domino serviatis, et, cum pro negotiis Ecclesiae vestrae ad curiam veneritis, honestum hospitium habeatis; \textit{Willelmus Malmesburiensis (c. 1095 – c. 1143), De gestis pontificum Anglorum, 179, 1506C:} Nec vero alicubi in Anglia diu fovere hospitium putabat honestum; \textit{Caffarus (c. 1080-c. 1166), Annales Ianuenses, 3:} tunc enim ipsum dominum Ihesum Christum collegerunt, quando uice eius palatia et hospitia honestissima atque inmensa stipendia domino apostolico suisque episcopis et cardinalibus sufficienter cum amore magno et tripudio impenderunt; \textit{Guillelmus Tyrensis (c. 1130–1186), Historia rerum gestarum in partibus transmarinis, 201, 0804A:} suis autem principibus singulis singula, honesta multum in urbe, non tamen longe a se, hospitia praeparari\dots\ Sed et suis nihilominus, non longe ab eodem palatio, honesta simul et commoda fecit hospitia praeparari; \textit{Odo de Soliaco (1150-1208), Synodicae constitutiones, 212, 0059A:} Praecipitur in eundo et redeundo a synodo honeste ambulent presbyteri, et honesta quaerant hospitia;  \textit{Aeneas Silvius Piccolomini (1405-1464), De duobus amantibus historia, 1, 1:} Plena illi semper est domus honestis hospitibus}}, 
in quae proborum uirorum \edtext{liberis}{\Afootnote{liberi \textit{va}}} uel necessitatis uel bonorum morum gratia diuersari licet. 

Accipiebat\ledsidenote{6} praeterea munera, non auaritiae, sed honoris \edtext{comparandaeque}{\Afootnote{comperandique \textit{Gd;} comparandique \textit{P m}}} \edtext{beniuolentiae}{\Afootnote{beneuolentię \textit{co;} beniuolencie \textit{C}}} gratia; quibus tamen in acceptandis ea lege utebatur ut multo ampliora rependeret quam acciperet. Testes sunt omnes qui hac \edtext{officiorum uicissitudine}{\Bendnote{\textit{Hilarius Pictaviensis, Tractatus super psalmos, 9, 0869D:} ut mare et terra et coelum Deum non tam voce, quam officiorum suorum vicissitudine atque observatione laudant; \textit{Symmach. ep. 1, 8a, 18, 5:} ut exemplo diligentiae tuae in officiorum vicissitudinem provocemur; \textit{Ioannes Saresberiensis (c. 1120-1180), Metalogicus, 199, 0827C:} nullum charitati, aut vicissitudini officiorum relinquit locum}} cum eo decertare uoluerunt. Complures in hoc coetu astare non dubito qui me uera praedicare nouerunt et qui post acceptos ab eo non paruos honores multa insuper dona tulerunt.

Vidi\ledsidenote{7} illum \edtext{quodam}{\Afootnote{quondam \textit{co}}} uesperi non sine graui stomacho 
\edtext{lacrimis \edtext{suffusum}{\Afootnote{suffusos \textit{C}}} \edtext{oculos}{\Afootnote{\textit{om. ve}}}}{\lemma{lacrimis\dots\ oculos}\Bendnote{\textit{Verg. A. 1, 223:} illum talis iactantem pectore curas / tristior et lacrimis oculos suffusa nitentis / adloquitur Venus; \textit{Sedul. Carmen de incarnatione, 19, 0775B:} illa pavens oculos suffusa nitentes, / Suspirans, imoque trahens a pectore vocem, Virgo refert; \textit{Ps.-Hegesippus, Historiae libri 5, p2, 8:} Cuius accepto indicio suffusus oculos lacrimis et quantum erat in eo uirtutis ingemescens ait; \textit{Abbo Floriacensis (c. 945–1004), Vita S. Eadmundi, 139, 0507C:} Quibus fatebaris oculos suffusus lacrymis quod eam iunior didicisses a quodam sene decrepito; \textit{Petrus Damianus (1007-1072), Apologeticum de contemptu saeculi, 145, 0273C:} Actionum quippe saecularium fumo suffusi oculos redeunt; \textit{Guibertus S. Mariae de Novigento (c. 1055–1124), De vita sua, 156, 0847C:} turbulenta et aestuans, et oculos moerore suffusa; \textit{Thomas Cantuariensis (c. 1118–1170), Epistolae, 190, 0711B:} Auditis siquidem ille quae in litteris continebantur, lacrymis suffusus oculos, post singultus et multas de moribus et conversatione Censorini nostri querimonias\dots}} \edtext{et}{\Afootnote{te \textit{Gd P m} }} 
cum multa indignatione Deum optimum maximumque testem citare atque, cum nemo cogeret minimeque ea re opus esset, \edtext{dirissima}{\Afootnote{durissima \textit{C}}} quaeque in se caputque suum \edtext{impetrantem}{\Afootnote{imprecantem \textit{co}}}, si ullas pecunias aut ulla 
\edtext{Symoniacae \edtext{perfidiae}{\Afootnote{perfidiaeque \textit{R}}} \edtext{praemia}{\Afootnote{praemis \textit{V Ge P}}} }{\Bendnote{\textit{Liv. 31, 2:} hodie subtracta ex condicionibus pacis praemia perfidiae habeant; \textit{Liv. 32, 15:} Non tamen inquit, T. Quincti, par perfidiae praemium est; \textit{Tac. A. 15, 54:} cum secum servilis animus praemia perfidiae reputavit; \textit{Ambros. Epistolarum classis I, 16, 1127C:} cum iam dignum praemium retulerint illi perfidiae suae; \textit{Beda (672-735), Historiam ecclesiasticam gentis Anglorum, 5, 21; 85:} simoniacam tamen perfidiam tota mente detestor ac respuo; \textit{Gregorius VII (c. 1020–1085), Registrum, 148, 0442C:} praedictam Ecclesiam Simoniacae perfidiae haeretica pravitate subversus invaserat; \textit{Thomas Aquinas (1225-1274), Catena aurea, in Mt., 10, 2; 56:} Hoc autem dicit ne Iudas, qui loculos habebat, de praedicta potestate pecuniam congregare vellet, damnans etiam hic perfidiam simoniacae haereseos; \textit{Francesco Petrarca (1304-1374), Africa, 6; 294:} Non tamen a nobis modo premia digna feretis / Perfidie; \textit{Petrarca, Africa, 7; 147:} Et michi, perfidie ne premia forte reportent, Est animus; \textit{Petrarca, De viris illustribus, p22, 9; 37:} quod, quasi perfidie premium sperantes, post primam turbatam secundam multo faciliorem pacem peterent}} 
\edtext{abs}{\Afootnote{ab \textit{pa}}} quoquam eorum accepisset, qui nuper in hunc sacrosanctum senatum apostolicum lecti noscuntur, seque ab \edtext{inuidis}{\Afootnote{infidiis \textit{V P m;} inuidiis \textit{Gd}}} atque malignis impie ac flagitiose eius criminis insimulari persancte iurabat. Angi eum uehementer et summis animi cruciatibus torqueri uideres quod sibi, ut dicebat, per detrahentium liuorem non liceret in uiros praestantes \edtext{ac}{\Afootnote{ab \textit{P;} ad \textit{m}}} bene meritos officiosum esse.


Vbi\ledsidenote{8} nunc sunt 
\edtext{rubiginosa illa \edtext{maliuolorum}{\Afootnote{\textit{sic omnes libri et codices}}} pectora}{\lemma{rubiginosa\dots\ pectora}\Bendnote{\textit{Ennod. carm. 1, 7, praef.:} adiacet dictis locupletibus, quantum sentio, rubiginosos lingua et sensu hebetes peritiae lima conponere; \textit{Alanus de Insulis (1128-1203), De planctu naturae, 210, 0468B:} invidia, quae continuae detractionis rubiginosa demorsione hominum animos demolitur}}, 
ubi sunt dirissimo felle manantia praecordia, ubi rabidorum 
\edtext{canum pestiferi dentes}{\lemma{canum\dots\ dentes}\Bendnote{\textit{Isid. Synonyma de lamentatione animae peccatricis, 83, 0857B:} Canum mos est latrare, canum linguas exerere, canum dente pestifero mordere, soli canes obtrectare noverunt}}, 
qui inexpiabili temeritate ausi sunt 
\edtext{os suum in caelo ponere}{\lemma{os\dots\ ponere}\Bfootnote{\textit{Cf. Ps 72 (73), 9:} Posuerunt in caelum os suum, et lingua eorum transivit in terra}\Bendnote{\textit{Ambros. De interpellatione Iob et David [CSEL], 3, 12; 2:} ponere in caelo os suum quid sit docet nos ille ex fratribus adulescentior; \textit{Paulin. Nol. Epistulae [CSEL], 12, 5; 1:} frustra posuerunt in caelo os suum; \textit{Salvianus Massiliensis (c. 400 – c. 470), De gubernatione Dei [CSEL], 4, 40; 2:} posuerunt enim, sicut de impiis scriptum est, posuerunt in caelo os suum, et lingua eorum transiit super terram; \textit{Petrus Damianus (1007–1072), Epistulae; 3999:} Desinat, oro, iam desinat quisquis ponit os suum in caelo, ut lingua eius transeat super terram}} 
et uicarii Christi fratrumque eius, tam excellentissimorum patrum, factum \edtext{damnare}{\Afootnote{dampnare \textit{pa}}}, atque illos praestantissimos uiros, qui prius a \edtext{domino}{\Afootnote{donimo \textit{R}}} electi sunt, 
\edtext{a quo sunt omnium regnorum potestates}{\lemma{a quo\dots\ potestates}\Bfootnote{\textit{Cf. Rom 13, 1:} non est enim potestas nisi a Deo quae autem sunt a Deo ordinatae sunt}\Bendnote{\textit{Greg. Liber sacramentorum, 78, 0205D:} Domine sancte, Pater omnipotens, aeterne Deus: sub cuius potestatis arbitrio omnium regnorum continetur potestas; \textit{Innocentius II (m. 1143), Epistolae et privilegia, 179, 0303B:} Rex regum et Dominus dominantium, in cuius manu sunt omnium potestates et omnia iura regnorum, ex incomprehensibili supernae providentiae dispensatione, quando vult, mutat tempora et transfert regna; \textit{Coelestinus III (c. 1106–1198), Epistolae et privilegia, 206, 1090A:} ad eum vias tuas ex toto dirigere, in cuius ditione sunt omnium potestates et iura regnorum, per quem reges regnant, dominantur principes, qui regibus dat salutem et imperatores via regia incedere, quos facit feliciter imperare; \textit{Iohannes Porta de Annoniaco (fl. c. 1350), Liber de coronatione Karoli IV imperatoris, 5, 29; 16:} ex quibus prefatum negotium sive causa tamquam maxima domino pape, patri patrum, noscitur reservata, in cuius manu sunt omnes potestates et omnia iura regnorum, quia in manu eius sunt omnes fines terre, et altitudines montium ipsius sunt}}, 
quam a summo pontifice declarati, perditissima audacia lacerare? \edtext{Caue, caue}{\Afootnote{\textit{alterum} caue \textit{om. Ge C Gd va o}}} tibi, lingua dolosa, ne 
\edtext{Deus \edtext{destruat}{\Afootnote{destruet \textit{pa}}} te in finem, \edtext{euellet}{\Afootnote{\textit{corr. scriba ipse in} euellat \textit{ve}}} te, et emigret te de tabernaculo tuo, et radicem tuam de terra uiuentium}{\lemma{Deus\dots\ uiuentium}\Bfootnote{\textit{Cf. Ps 52, 5 (51, 7)}: sed Deus destruet te in sempiternum terrebit et euellet te de tabernaculo et eradicabit te de terra uiuentium semper \textit{(in versione antiqua:} Deus destruet te in finem, euellat te, et emigret te de tabernaculo suo, et radicem tuam de terra uiuentium)}}! 
Sed haec illi \edtext{uiderunt}{\Afootnote{uiderint \textit{ve}}} qui \edtext{fecerunt linguam suam nouaculam acutam}{\lemma{fecerunt\dots\ acutam}\Bfootnote{\textit{Cf. Ps 52, 2 (51, 4):} insidias cogitat lingua tua quasi nouacula acuta faciens dolum}} et \edtext{sedentes aduersus fratrem suum \edtext{tota}{\Afootnote{\textit{om. va}}} die concinnabant dolos}{\lemma{sedentes\dots\ concinnabant dolos}\Bfootnote{\textit{Cf. Ps 49, 19-20:} Os tuum abundavit malitia, et lingua tua concinnabat dolos. Sedens adversus fratrem tuum loquebaris}\Bendnote{\textit{Ambr. Expositio in psalmum David CXVIII, 15, 1389B:} Os tuum abundavit nequitia, et lingua tua concinnabat dolos. Sedens adversus fratrem tuum detrahebas; \textit{Arnobius iunior (fl. 460), Commentarii in Psalmos, 53, 0398D:} Sicut novacula acuta faciens dolum, tanto acumine concinnas dolos}}. Sed haec illi uiderunt; nos interea reliquas Petri uirtutes, \edtext{non quibus debemus, sed quibus possumus}{\lemma{non\dots\ debemus\dots\ possumus}\Bendnote{\textit{Rhet. Her. 4, 4:} Non enim, cum dicimus esse exornationem (\dots) et sumimus hoc exemplum a Crasso: ``quibus possumus et debemus'' testimonium conlocamus, sed exemplum. \textit{Cic. Paradoxa stoicorum Paradoxon 5:} `Quibus et possumus et debemus.' Nos vero, siquidem animo excelso et alto et virtutibus exaggerato sumus, nec debemus nec possumus; tu posse te dicito, quoniam quidem potes, debere ne dixeris, quoniam nihil quisquam debet, nisi quod est turpe non reddere.}}, laudibus \edtext{prosequamur}{\Afootnote{p-sequamur \textit{V C P}}}.

Itaque,\ledsidenote{9} ut propositus ordo postulat, nunc ipsius prudentiam contemplemur, quamuis eius excellentia iam per ea, quae narrauimus, magna ex parte potuit esse manifesta. Quamobrem tantum eam \edtext{partem}{\Afootnote{patrem \textit{Ge}}} \edtext{attingam}{\Afootnote{attingat \textit{C}}} qua ita \edtext{sese}{\Afootnote{se se \textit{co}}} inter principes Christianos gessit ut, cum unius erga \edtext{eum}{\Afootnote{\textit{om. R}}} \edtext{beniuolentiam}{\Afootnote{\textit{sic omnes libri et codices praeter} beneuolentiam \textit{co}}} consideres, credas ab aliis minime dilectum. \edtext{Lex quippe amicitiae ita habet ut amicos inimicorum minime diligamus}{\lemma{Lex\dots\ diligamus}\Bfootnote{\textit{Cf. Aug.  Sermones de Scripturis, 49, 6, 6:} Incipiunt esse de tribus amicis duo inter se inimici, quid faciat medius qui remansit? Vult, exigit, flagitat a te ut oderis cum illo quem odisse coepit, et haec verba tibi dicit: Non es amicus meus, quia es amicus inimici mei.}}; hic tamen sua prudentia consecutus est ut aeque carus omnibus haberetur, nec ullus esset qui eius amicitiam ultro non expeteret, et adeptam studiis omnibus non coleret atque foueret. Ostendit id nouissima haec ipsius legatio, in qua \edtext{omnes}{\Afootnote{habes \textit{o}}} Italiae populi singulique ipsorum principes summis decertarunt studiis quisnam eum amplioribus honoribus exciperet et prosequeretur, illeque se uicisse putabat qui plurima in eum ornamenta contulisset. In quibus acceptandis \edtext{adeo}{\Afootnote{a Deo \textit{R}}} prudens temperamentum inibat ut, cum omnium communis esset amicus, singuli tamen sibi eum \edtext{uendicasse}{\Afootnote{\textit{sic omnes libri et codices}}} putabant. Vnde sua illi negotia credebant et rerum omnium summam fidei ipsius ultro committebant. Fallebat neminem et, \edtext{communem rem gerendo, singulorum tamen uidebatur aduocatus}{\Bendnote{\textit{Cf. Cic. Clu. 69, 7:} Summa est enim, iudices, hominis in communem municipi rem diligentia, in singulos municipes benignitas, in omnis homines iustitia et fides; \textit{Cic. de fin. 3, 64:} ut enim leges omnium salutem singulorum saluti anteponunt, sic vir bonus et sapiens et legibus parens et civilis officii non ignarus utilitati omnium plus quam unius alicuius aut suae consulit.}}. 

Secedebat\ledsidenote{10} bis terque per diem in cubiculum aut in aliquem secretiorem locum, in quo ad multam horam \edtext{deambulando}{\Afootnote{deamulando \textit{pa}}} summas reipublicae Christianae rationes tacitus secum computabat; totas enim \edtext{animi}{\Afootnote{anni \textit{ve}}} uires, postquam a legatione redierat, in pacem Italiae et perfidi hostis \edtext{Christiani}{\Afootnote{Christiane \textit{pa}}} \edtext{exitum}{\Afootnote{\textit{Ge R C Ge P m va o pa;} exitium \textit{V ve co}}} intenderat; id unum \edtext{moliebatur}{\Afootnote{muliebatur \textit{V P}}}, id parabat, illuc omnia sua studia conuerterat, fecissetque uotis satis \edtext{ni}{\Afootnote{in \textit{Ge}}} eum nobis haec dira atque crudelis mors tam repente praeripuisset. Vincebat ingenio humana consilia et tantum grauioribus reipublicae curis natus \edtext{uidebatur}{\Afootnote{\textit{om. R}}}; haec erat praecipua eius \edtext{uoluptas}{\Afootnote{uoluntas \textit{co}}} a qua nullis aliis \edtext{oblectamentis poterat diuelli; huius delectabatur gustu, huius solius escis pascebatur}{\lemma{oblectamentis\dots\ pascebatur}\Bendnote{\textit{Cic. Pis. 50, 20, 1:} His ego rebus pascor, his delector, his perfruor; \textit{Ambr. Epistolarum classis I, 16, 1048B:} In his dives est, in his pascitur, in his delectatur, quasi in omnibus divitiis; \textit{Aug. Sermo II de symbolo, 40, 0639:} His igitur oblectamentis mens delectetur, pascatur anima christiana; \textit{Gregor. I Moralia 75, 0833A:} gustu incircumscripti luminis pascitur; \textit{Aelredus Rievallensis  (1110–1167), Sermones de tempore, 195, 0272C:} Et quid erant illi boni discipuli eius, nisi esca eius, in quibus ipse delectabatur, quorum dilectione et devotione pascebatur?}}. Omnium saluti die noctuque inseruiebat, et tamen a nonnullis negligentiae \edtext{accusabatur}{\Afootnote{accusabantur \textit{Gd P}}}; quin tamquam \edtext{superbum \edtext{difficilemque}{\Afootnote{difficlemque \textit{V}}}}{\Bendnote{\textit{Cic. Vat. 3, 5:} Scilicet aspera mea natura, difficilis aditus, gravis vultus, superba responsa, insolens vita; nemo consuetudinem meam, nemo humanitatem, nemo consilium, nemo auxilium requirebat; \textit{Stat. Theb. 2, 345-347:} atque illum sollers deprendere semper / fama duces tumidum narrat raptoque superbum / difficilemque tibi; \textit{Ennod. Vita B. Antonii, 63, 0241C:} Sed quam superbus est magnitudine, tam difficilis ascensu}} ingrati criminabantur, cum tamen et mitissimus esset et \edtext{facillimus}{\Afootnote{facillissimus \textit{va}}}. \edtext{Nouit hoc tantus domesticorum eius numerus, norunt}{\lemma{Nouit\dots\ norunt}\Bendnote{\textit{Herbertus de Boseham (m. 1186), Vita S. Thomae, 190, 1226A:} Novit enim dominus noster rex Francorum, qui hic, norunt et hi qui astant, novit et mundus, et testantur opera, cum adhuc essem in aula, qualem me in illo officio aulico ad vestram utilitatem et honorem exhibuerim.}} amici et alii omnes, qui eius \edtext{familiaritate usi fuerunt}{\Bendnote{\textit{Ps. Probus, Vita Vergilii, 1.4:} insigni concordia et familiaritate usus Quintilii Tuccae et Vari; \textit{Gerbertus Auriliacensis (940-1003), Epistolae scriptae ante summum pontificatum, 139, 0239D:} Nam amici, qui familiaritate beati patris Adalberonis mecum usi fuerunt, mecumque laborabant}}, quibus semper, cum per publicas licuisset curas, placidum \edtext{sese}{\Afootnote{se se \textit{co}}} exhibebat, affabilem, comem, benignum, ut socium crederes, non dominum.

Tanta\ledsidenote{11} uero animi moderatione erat ut irasceretur perraro, iratus autem \edtext{extemplo}{\Afootnote{exemplo \textit{Ge C P m o}}} \edtext{animum ad tranquillitatem reuocaret}{\lemma{animum\dots\ reuocaret}\Bendnote{\textit{Lactant. Divin. instit. 6, 6, 0702C:} Haec perturbatum ac fluctuantem animum ad tranquillitatem suam revocat: haec mitigat, haec hominem sibi reddit; \textit{Petrus Blesensis (1135-1212), Sermones, 207, 0575C:} Puer Iesus raro irascitur; iratus vero de levi placatur}}. \edtext{De nullo obloquebatur, detrahebat nemini}{\lemma{De nullo\dots\ nemini}\Bendnote{\textit{Hilarius Pictaviensis, Tractatus super psalmos, 9, 0476D:} Ipsi enim qui in portis orandi victus causa sedeant, obloquuntur et detrahunt: et in eo ebrii psallunt; \textit{Cassiod. Expositio in Psalterium, 70, 0046A:} detrahentium verba mordacia, qui potestati divinae nefandis dogmatibus obloquuntur; \textit{Thomas Aquinas, Summa theologiae, II-II, 73, 2; 20:} Unde ille, per se loquendo, detrahit qui ad hoc de aliquo obloquitur, eo absente, ut eius famam denigret}}, et, nisi familiarium suorum, aliorum \edtext{uitae minime erat curiosus}{\lemma{uitae\dots\ curiosus}\Bendnote{\textit{Aug. de civ. Dei 41, 0306:} ut talium quoque rerum quasi peritus appareas, et placeas illicitarum artium curiosis, vel ad eas facias ipse curiosos; \textit{Isid. Synonyma de lamentatione animae peccatricis, 83, 0857B:} Noli quaerere quid quisque dicat vel faciat; devita curiositatem, omitte curas alienae vitae, omitte curam quae ad causam tuam non pertinet; \textit{Ivo Carnotensis (c. 1040–1115), Epistolae, 162, 0126D:} Tanto minus itaque iste alienae vitae curiosus inspector et suae desidiosus corrector audiendus est; \textit{Thomas Aquinas (1225-1274), Catena aurea, in Lc., 21, 8; 16:} Curiositas autem huius vitae quamvis nihil inhibitorum continere videatur, si tamen ad cultum divinum non coadiuvet, vitanda est}}. Maximum eius conuicium putabatur si quem, quod tamen parcissime fiebat, per iocum aliquo \edtext{\edtext{urbanitatis}{\Afootnote{urbanitate \textit{pa}}} sale respersisset}{\lemma{urbanitatis\dots\ respersisset}\Bendnote{\textit{Cic. de or. 1, 137, 159; 22:} libandus est etiam ex omni genere urbanitatis facetiarum quidam lepos, quo tamquam sale perspergatur omnis oratio; \textit{Cic. de or. 2, 182, 231; 124:} explicare nobis totum genus hoc iocandi quale sit et unde ducatur; praesertim cum tantam vim et utilitatem salis et urbanitatis esse fateatur; \textit{Alcuinus (730-804), Epistolae, 100, 0260A:} quidquid urbanitatis sale conditum cognoscitur, vestris intellectualibus auribus favorabile; \textit{Ruricius (c. 435 - c. 510), Epistulae [CSEL], 1, 4; 1:} Recepi apices unianimitatis tuae tam gratia quam eloquentia, tam amore pariter quam lepore, tam sale quam melle respersas, in quibus nec dulcedini deesset aliquid nec sapori; \textit{Flodoardus Remensis (894–966), Historia ecclesiae Remensis, 135, 0279C:} diversa reperitur direxisse scripta, pio sale respersa, et divinis auctoritatibus referta; \textit{Petrus Cellensis (c. 1115-1183), Epistolae, 202, 0605D:} Nihil fetidum, nihil foedum, nihil emortuum vivus ille sermo tuus, nitidus, et sale sufficienter respersus continet; \textit{Francesco Petrarca (1304-1374), Epistole familiares, 1, 10; 6:} haud absimilis est seni plautino, cuius mores atque animum mordaci sale respergens, servus ille in Aulularia\dots}}.

Cibi\ledsidenote{12} uinique \edtext{moderatissimi}{\Afootnote{moderatissimus \textit{ve}}}, somnolentiae nullius, immo uero peruigil et qui multis quotidie horis auroram solitus esset praeoccupare; quod totum tempus usque ad solis exortum \edtext{grauioribus}{\Afootnote{grouioribus \textit{P}}} reipublicae cogitationibus impendere consueuerat, quae \edtext{tales}{\Afootnote{quales \textit{m}}} menti eius assidue obuersabantur quales mortalium animis uix illabi posse putares. In cognatos \edtext{ac}{\Afootnote{et \textit{m}}} necessarios nequaquam prodigus, immo uero maxime parcus, excepto hoc piissimo fratre comite Hieronymo; quem quoniam ab inclito duce \edtext{Mediolani}{\Afootnote{Mediolanensi \textit{m}}} \edtext{connubio}{\Afootnote{connubia \textit{Gd P m}}} filiae dignatum \edtext{cernebat}{\Afootnote{retinebat \textit{va}}}, uoluit \edtext{fraterna}{\Afootnote{frater ita \textit{va}}} munificentia illustrare. Qua in re tale temperamentum adhibuit ut id cum ingenti honore ac laude sedis \edtext{apostolicae}{\Afootnote{appostolice \textit{C Gd}}} contingeret. Vrbem \edtext{enim}{\Afootnote{\textit{om. va}}} Imolam, quae iam praefecti eius culpa ad alios \edtext{deuenerat}{\Afootnote{deuenetra \textit{R}}}, quadraginta milibus ducatorum de propriis facultatibus redemptam \edtext{imperio}{\Afootnote{\textit{om. R}}} ecclesiae restituit. Cuius exaugendi tanto flagrabat studio ut ne \edtext{minimam}{\Afootnote{nimiam \textit{Ge}}} quidem eius particulam deperire pateretur. \edtext{Vnde}{\Afootnote{Vade \textit{Ge o}}} urbem illam magis ecclesiae quam fratris gratia uoluit \edtext{uendicare}{\Afootnote{\textit{sic omnes libri et codices}}}.

Porro\ledsidenote{13} iustitiam ipsius ex illo spectare licet quod in tanta \edtext{rerum}{\Afootnote{rerum rerum \textit{va}}} potestate constitutus nemini uim attulit, nullum uiolenter oppressit. Vnde et uicario illi Imolae, quamuis de ecclesia male merito, non solum uictum, sed nihil prioribus minores fortunas dari curauit. Quattuordecim \edtext{enim}{\Afootnote{\textit{om. pa}}} milia ducatorum accepit \edtext{et}{\Afootnote{ec \textit{V;} hec \textit{P m}}} \edtext{opulentissimum}{\Afootnote{opulenlentissimum \textit{C}}} oppidum Bosti, ex quo et aliis a duce adiectis praediis plus quam quinque milia ducatorum quotannis capere poterit, neque Imolam nisi eo uolente redimere uoluit. Magistratus hortabatur ius suum absque ullo respectu cuique administrare. Et licet \edtext{nonnumquam}{\Afootnote{nonniquam \textit{co}}}, domesticorum amicorumque \edtext{euictus}{\Afootnote{euectus \textit{pa}}} precibus, aut litteris aut nuntiis multos iudicibus commendaret, id tamen \edtext{citra}{\Afootnote{circa \textit{V Gd P;} circa al. citra \textit{m}}} \edtext{cuiusque}{\Afootnote{cuiusq-m \textit{Gd}}} iniuriam fieri uolebat. Vnde et cum \edtext{a}{\Afootnote{o \textit{P}}} gubernatore Vrbis aliquando interrogaretur quidnam de illis \edtext{fieri uellet}{\Afootnote{\textit{om. V Gd P m ve co pa}}} qui \edtext{iniustam}{\Afootnote{in iustam \textit{P}}} causam fouentes tamen ipsius nomine commendabantur, ``Illud'' inquit ``ut iustitiam mearum precum gratia minime \edtext{uioles}{\Afootnote{uiolet \textit{P m}}}, nec secus feceris etiam si te germanus meus \edtext{Hieronymus}{\Afootnote{Hironimus \textit{P}}} nomine meo aliud precabitur''.  Hoc ipsum et senatori Vrbis \edtext{interrogatus respondit}{\Afootnote{respondit interrogatus \textit{va}}}, hoc gubernatoribus, hoc praefectis, hoc omnibus legationis suae iudicibus saepe mandabat.  Dicebat enim se amicis operam suam rogantibus negare non posse, sed illius causa non nisi iustum honestumque fieri permaxime uelle.  \edtext{Hinc}{\Afootnote{Hic \textit{C}}} et rescripta non nisi \edtext{sanctissima}{\Afootnote{sanctassima \textit{pa}}} faciebat \edtext{atque}{\Afootnote{adque \textit{P}}}, ne cui ministrorum peccandi daretur occasio, omnium decretorum suorum, similiter et litterarum, exemplaria apud notarios extare iusserat.  Huius praeclarissimae uirtutis nec moriens obliuisci potuit; nam, cum hortaretur ab amicis testamentum condere, ``Nihil'', inquit, ``meum habeo; omnia sunt ecclesiae. Supplicabitis tamen meo nomine summo pontifici ut aes alienum, cuius maiorem partem pro \edtext{redimendis}{\Afootnote{\textit{om. R}}} ecclesiae rebus contraxi, pro sua benignitate dissoluat.''

Cogor\ledsidenote{14} hoc loco potissimas eius \edtext{praetermittere}{\Afootnote{premittere \textit{m}}} laudes, \edtext{cum}{\Afootnote{\textit{om. R}}} temporis exclusus angustia, \edtext{tum}{\Afootnote{t-n \textit{Gd}}} \edtext{ipsarum}{\Afootnote{ipsa \textit{co}}} rerum multitudine superatus.

\endnumbering

\newpage

\section*{Loci similes}
\doendnotes{B}


\end{document}
officiorum uicissitudine: \Bendnote{\textit{Hilarius Pictaviensis, Tractatus super psalmos, 9, 0869D:} ut mare et terra et coelum Deum non tam voce, quam officiorum suorum vicissitudine atque observatione laudant; \textit{Symmach. ep. 1, 8a, 18, 5:} ut exemplo diligentiae tuae in officiorum vicissitudinem provocemur; \textit{Ioannes Saresberiensis (c. 1120–1180), Metalogicus, 199, 0827C:} omnia liberalia studia convellit, omnem totius philosophiae impugnat operam, societatis humanae foedus distrahit, et nullum charitati, aut vicissitudini officiorum relinquit locum}
