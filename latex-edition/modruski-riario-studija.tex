%-*-coding:utf-8;-*-
%&LaTeX
\documentclass[a5paper,twoside]{article}
\usepackage{polyglossia}
\setmainlanguage{croatian}
\setotherlanguage{latin}

\defaultfontfeatures{Ligatures=TeX}

\usepackage{indentfirst,titling,enumitem}

% prilagodba dimenzija stranice
\usepackage[a5paper,pdftex]{geometry}%a4paper,pdftex
%\usepackage[a4paper,pdftex]{geometry}
\geometry{scale=1,paperwidth=161mm,paperheight=230mm,
bindingoffset=0in,top=27mm,inner=21mm,outer=26mm,width=117mm,height=180mm}%
\parindent=7mm

%\usepackage[small,center,rm]{titlesec}
%\titlespacing{\section}{0pt}{11mm}{7mm}

\usepackage[medium,center,rm]{titlesec}
\titlelabel{\thesection.\enspace}
\titlelabel{\thesubsection.\enspace}
% \chapter, \subsection...: no additional code
\titleformat{\section}
  {\normalfont\large\bfseries\uppercase}
  {\thesection.\enspace}{0em}{}
\titlespacing{\section}
  {7mm}{3.5ex plus .1ex minus .2ex}{1.5ex minus .1ex}
\titleformat{\subsection}
{\normalfont\large\bfseries}
{\thesubsection.\enspace}{0em}{}
\titlespacing{\subsection}
  {7mm}{3.5ex plus .1ex minus .2ex}{1.5ex minus .1ex}


% titling
\pretitle{\begin{center}\vskip 5mm}
\posttitle{\par\end{center}}
\preauthor{}
\postauthor{}
\predate{}
\postdate{}

% slova
\setmainfont{Times New Roman}
%\setsansfont{Old Standard TT}

% makro za regum gesta
\newcommand{\kratica}{\emph{Regum gesta}}%

% reference bold
\newcommand{\refb}[1]{\textbf{\ref{#1}}}

\renewcommand{\labelenumi}{\Alph{enumi}}

%tekuće glave
%tekuće glave
\usepackage{fancyhdr}
\pagestyle{fancy}
\fancyhead{} % clearall
\fancyhead[RO,LE]{\thepage}
\fancyhead[EC]{\MakeUppercase{Nikola Modruški: Djela u službi pape Siksta IV.}}
\fancyhead[OC]{\MakeUppercase{Govor za Pietra Riarija (1474)}}
\fancyfoot{}
\renewcommand{\headrulewidth}{0.4pt}

\makeatletter
\renewcommand{\@makefnmark}{\mbox{%
\textsuperscript{\normalfont\@thefnmark\ }}}
\makeatother

\renewcommand\footnotesize{\fontsize{10}{10.5} \selectfont}
\renewcommand\tiny{\fontsize{9}{9.5} \selectfont}
\renewcommand\large{\fontsize{11}{11.5} \selectfont}
\renewcommand\Large{\fontsize{12.5}{13} \selectfont}

% fusnote i crta
\renewcommand{\footnoterule}{\vspace*{-7pt}
    \noindent\rule{2in}{0.4pt}\vspace*{6.6pt}}

\setlength{\headheight}{12.6pt}

% makro za naziv programa
\newcommand{\makeup}[1]{\MakeUppercase{\emph{#1}}}


% prilagodba quote
% prilagođavamo citat propozicijama
\renewenvironment{quote}
               {\list{}{\rightmargin 0mm
                \leftmargin 7mm
                \itemindent 0em}%
                \item\relax}
               {\endlist}

\newenvironment{opisi}{%
\let\olditem\item% 
\renewcommand\item[2][]{\olditem ##1 ##2 \itemsep 3pt}%
\begin{description}}{\end{description}%
}


%siročići i udovice
% \widowpenalty=10000
% \clubpenalty=10000

\begin{document}
%%tth:\begin{html}<META HTTP-EQUIV="CONTENT-TYPE" CONTENT="text/html; charset=utf-8">\end{html}

\hyphenation{his-to-ri-o-graf-ske po-li-tič-ko po-li-tič-ko-his-to-ri-o-graf-ske}
\hyphenation{quae-sti-o-nes quin-qua-gin-ta quo-que DXXXVIII Zuo-ni-me-ro}
\hyphenation{quad-ra-ta quo-mo-do Schwand-tne-ro-vi cro-a-tiae I-su-krs-ta}
\hyphenation{Pas-qua-li-go XVIII U-di-sla-va in-tel-le-xis-set kra-ljev-stvo}
\hyphenation{in-ua-se-runt quod pes-si-mo pe-dan-tno-an-ti-kvar-ski pas-cha-les}
\hyphenation{Pa-lu-šom MDCCXLVIII}

% No extra spacing after periods
\frenchspacing

% Font sizing
\fontsize{11}{13.2}
\selectfont

% veća čitljivost
\linespread{1.1}

%uvlaka paragrafa
\setlength{\parindent}{7mm}

% počinje na s. 5
\setcounter{page}{169}

\title{\Large{\MakeUppercase{Govor za Pietra Riarija (1474)}}}

\date{}%Datum ove verzije: \today}

\maketitle 

%\texttt{Završna verzija, uredniku na čitanje.}
\thispagestyle{empty}
% [ova stranica namjerno prazna]

\section{Uvodne napomene}

Latinski nadgrobni govor modruškog biskupa Nikole za kardinala Pietra Riarija (preminulog 5. siječnja, sahranjenog 18. siječnja 1474. u Rimu) igrom je slučaja prva poznata nam tiskana knjiga nekog hrvatskog autora, svojevrstan latinski pandan devet godina kasnije tiskanog glagoljskog \textit{Misala po zakonu Rimskoga dvora.}

Autograf djela nije sačuvan. Priređujući kritičko izdanje i prvi hrvatski prijevod Nikolina govora,\footnote{Znanstveno je izdanje, kako ga je priredio autor ove studije, dostupno od 2009.\ u digitalnoj zbirci \textit{\textlatin{Croatiae auctores Latini}}; prema tom izdanju govor je nedavno preveden na engleski: Brendan Cook i Jennifer Mara DeSilva, »Princely Ambiguity: A Translation of Nikolaus of Modruš’ Funeral Oration for Cardinal Pietro Riario: Oratio in funere Petri Cardinalis Sancti Sixti (1474)«, \textit{Royal Studies Journal,} 5 (2018), 2, str.~92–128.} željeli smo uspostaviti najbolji mogući tekst, omogućiti uvid u tekstualne inačice trinaest svjedoka predaje, te djelo učiniti današnjem čitaocu jezično i stvarno što razumljivijim.

Boljem razumijevanju namijenjen je i ovaj uvod. Predstavit ćemo najprije povijesni kontekst u kojem su se pojavljivala izdanja govora za Riarija, potom tematsku strukturu, žanrovsku pripadnost i stilska obilježja samog djela, da bismo na koncu prikazali svjedoke predaje teksta, njihov međusobni odnos, i načela ovog izdanja. 

\section{Povijesni kontekst}

\section{Tematska i retorička struktura govora}

Nikolin govor za Riarija ima jasnu i jednostavnu strukturu (\textit{\textlatin{dispositio}} antičke retorike): uvod, glavni dio i završetak. Strukturu ovdje dodatno tematski raščlanjujemo navodeći brojeve odlomaka u kritičkom izdanju (uz siglu MR) i dodatno citirajući početke i krajeve pojedinih cjelina. Napomene u zagradama standardni su retorički termini za pojedini dio govora ili pojedine ključne riječi.

\begin{description}[nolistsep,itemsep=3pt,font=\rmfamily]
\item[1. MR 1–2 Uvod \textlatin{(exordium, prooimion)}] – \textlatin{Cum in omni funebri celebratione\dots\ quam ingratitudinis notam subire uerebor.}
\item[2. MR 3–22 Pripovjedni dio (narratio)] Riarijev život, vrline, smrt: \textlatin{Dicturus igitur de laudibus reuerendissimi domini\dots\ ad dominum suum confestim euolauit.}
\begin{description}[nolistsep,itemsep=3pt,font=\rmfamily]
\item[2.1. MR 3] Domovina i rod: \textlatin{eas laudes, quas uel a parentibus uel a patria\dots\ omnibus futuris seculis non desinent celebrari.}
\item[2.2. MR 3] Vrline: \textlatin{Quae quidem tametsi satis grandis eius gloria sit\dots\ quae quales in eo fuerint, breuiter explicare conabor.}
\item[2.3. MR 4–5] Dječaštvo: \textlatin{Qua igitur pietate primum erga deum\dots\ aut dicere illi quare sic fecisti.}
\begin{description}[nolistsep,itemsep=3pt,font=\rmfamily]
\item[2.3.1. MR 4] Dječaštvo – stupanje u franjevački red i pobožnost \textlatin{(pietas): Qua igitur pietate primum erga deum\dots\ seruatisque pro more religionis rite caerimoniis uestem Christi induit.}
\item[2.3.2. MR 4] Dječaštvo – školovanje i mudrost \textlatin{(prudentia): Qua assumpta ita omnia tyrocinii rudimenta libens promptusque\dots\ ita memoriter recitabat ut ea illum heri aut nudiustertius memoriae mandasse existimares.}
\item[2.3.3. MR 5] Dječaštvo – dolazak u Rim i proročka moć \textlatin{(divinitas): Perfectis igitur quam celerrime omnium bonarum artium studiis\dots\ aut dicere illi quare sic fecisti.}
\end{description}
\item[2.4. MR 6–11] Kardinalska čast – dužnosti kardinala \textlatin{(officia): Petrus igitur per hunc modum in principatu constitutus\dots\ in uiros praestantes ac bene meritos officiosum esse.}
\begin{description}[nolistsep,itemsep=3pt,font=\rmfamily]
\item[2.4.1. MR 6] Dužnosti — darežljivost \textlatin{(munificentia): Turpe enim et indecorum merito ducebat\dots\ culpam magis dolens quam damnum.}
\item[2.4.2. MR 7] Dužnosti – ustrajnost \textlatin{(perseverantia): A suscepto semel negotio nullo metu\dots\ magnanimitatis eius testimonia.}
\item[2.4.3. MR 7] Dužnosti – milostivost \textlatin{(clementia): Sed illud mea sententia uincit uniuersa\dots\ ut inimicis suis benefacere gauderet.}
\item[2.4.4. MR 8] Dužnosti – iskrenost \textlatin{(sinceritas): Nihil in se fictum, nihil subdolum\dots\ sed integritate uiros superare.}
\item[2.4.5. MR 8–9] Dužnosti – kardinalski dvor \textlatin{(familia)} i darežljivost \textlatin{(munificentia et liberalitas): Sane munificentiae liberalitatisque eius\dots\ bonorum morum gratia diuersari licet.}
\item[2.4.6. MR 10] Dužnosti – primanje darova: \textlatin{Accipiebat praeterea munera, non auaritiae\dots\ multa insuper dona tulerunt.}
\item[2.4.7. MR 11] Dužnosti – optužbe za simoniju: \textlatin{Vidi illum quodam uesperi non sine graui stomacho\dots\ in uiros praestantes ac bene meritos officiosum esse.}
\end{description}
\item[2.5. MR 12] Patetični ekskurs protiv zavidnika: \textlatin{Vbi nunc sunt rubiginosa illa maliuolorum pectora\dots\ tota die concinnabant dolos.}
\item[2.6. MR 12–13] Povratak na glavnu temu (hipostrofa): \textlatin{Sed haec illi uiderunt\dots\ magna ex parte potuit esse manifesta.}
\item[2.7. MR 13–14] Mudrost (prudentia) na dužnosti: \textlatin{Quamobrem tantum eam partem attingam\dots\ ut socium crederes, non dominum.}
\begin{description}[nolistsep,itemsep=3pt,font=\rmfamily]
\item[2.7.1. MR 13–14] Mudrost u politici (diplomatska misija u Italiji, prihvaćanje »uresa«, zalaganje za mir u Italiji): \textlatin{Ostendit id nouissima haec ipsius legatio\dots\ ni eum nobis haec dira atque crudelis mors tam repente praeripuisset.}
\item[2.7.2. MR 14] Zauzetost za opće dobro \textlatin{(omnium salus): Vincebat ingenio humana consilia\dots\ ut socium crederes, non dominum.}
\end{description}
\item[2.8. MR 15–16] Umjerenost \textlatin{(moderatio): Tanta uero animi moderatione erat\dots\ urbem illam magis ecclesiae quam fratris gratia uoluit uendicare.}
\item[2.9. MR 17] Pravičnost \textlatin{(iustitia): Porro iustitiam ipsius ex illo spectare licet\dots\ pro sua benignitate dissoluat.}
\begin{description}[nolistsep,itemsep=3pt,font=\rmfamily]
\item[2.9.1. MR 17] Pravičnost pri povratu grada Imole: \textlatin{Porro iustitiam ipsius ex illo spectare licet\dots\ neque Imolam nisi eo uolente redimere uoluit.}
\item[2.9.2. MR 17] Pravičnost u ophođenju s dužnosnicima: \textlatin{Magistratus hortabatur ius suum absque ullo respectu\dots\ non nisi iustum honestumque fieri permaxime uelle.}
\item[2.9.3. MR 17] Pravičnost u izdavanju isprava: \textlatin{Hinc et rescripta non nisi sanctissima\dots\ apud notarios exstare iusserat.}
\item[2.9.4. MR 17] Pravičnost u sastavljanju oporuke: \textlatin{Huius praeclarissimae uirtutis nec moriens\dots\ pro sua benignitate dissoluat.}
\end{description}
\item[2.10. MR 18] Izostavljanje \textlatin{(praetermissio): Cogor hoc loco potissimas eius praetermittere\dots\ felicem meritorum ipsius copiam pauca dicendo uitiare.}
\item[2.11. MR 18–19] Pobožnost kao posljednja spomenuta i posebna vrlina \textlatin{(pietas): Illud unum dicam, cunctis, qui eius consuetudinem nouerunt\dots\ omnium scientiarum libris egregie refertam.}
\begin{description}[nolistsep,itemsep=3pt,font=\rmfamily]
\item[2.11.1. MR 18] Pobožnost prema papi: \textlatin{Illud unum dicam, cunctis, qui eius consuetudinem nouerunt\dots\ nunc mortui desiderio adeo moueatur.}
\item[2.11.2. MR 19] Pobožnost pri obnavljanju i opremanju crkava i samostana: \textlatin{Finem dicendi faciam si prius illud summum eius pietatis munus\dots\ omnium scientiarum libris egregie refertam.}
\end{description}
\item[2.12. MR 19–22] Smrt: \textlatin{Sed dei uoluntate nobis tam repente ademptus est\dots\  ad dominum suum confestim euolauit.}
\begin{description}[nolistsep,itemsep=3pt,font=\rmfamily]
\item[2.12.1. MR 20] Neustrašivost i strpljivost u bolesti \textlatin{(patientia): Cuius miserationis dilectionisque certa indicia\dots\ dolores mira patientia pertulit.}
\item[2.12.2. MR 20] Ispovijest i pokajanje: \textlatin{Delicta, quae uel aetatis uel fortunae\dots\ diuinam uoluntatem accinctus praestolabatur.}
\item[2.12.3. MR 20–21] Govor pred smrt \textlatin{(novissima verba): Iamque uicinus morti\dots\ uel meo exemplo discite.}
\item[2.12.4. MR 22] Opis smrtnog časa: \textlatin{His atque aliis huiusmodi plerisque summa cum religione\dots\ ad dominum suum confestim euolauit.}
\end{description}
\end{description}
\item[3. MR 23 Završetak (peroratio)] pohvala i utjeha: \textlatin{O felix atque iterum felix\dots\ sit nomen domini benedictum. Amen.}
\end{description}

Ključ za razumijevanje strukture daje sam govornik. Već u sredini prve rečenice (MR 1) kao svrhu govora sugerira veličanje dostignuća i vrlina pokojnog prijatelja:

\begin{quote}
\begin{latin}
quo extremum amici munus rebus ab eo bene gestis uirtutumque ipsius copia ac splendore amplissimis laudibus exornarent
\end{latin}

posljednju počast prijatelju uveličati opsežnom pohvalom njegovih dostignuća te obilja i sjaja njegovih vrlina
\end{quote}

U posljednjim rečenicama MR 3, pak, govornik najavljuje kojih će točno šest duhovnih vrlina preminulog kardinala biti predmet hvale.

\begin{quote}
\begin{latin}
habeo tamen et alias immortales ac propemodum diuinas animi ipsius laudes (ut fortunae corporisque quaelibet ingentia bona tamquam aliena relinquam): pietatem, magnitudinem animi, munificentiam, prudentiam, modestiam, atque iustitiam; quae quales in eo fuerint, breuiter explicare conabor.
\end{latin}

ipak znam i druge besmrtne i gotovo božanske odlike njegova duha (da i ne spominjem dobra zadobivena srećom ili tjelesnim sposobnostima; ma kako ona bila silna, ipak su u stanovitoj mjeri tuđa): pobožnost, veličanstvenost duha, darežljivost, razboritost, skromnost i pravednost.  Kakve su one bile u pokojniku, pokušat ću kratko izložiti.
\end{quote}

Kao što se vidi iz pregleda strukture, upravo su vrline Riariju omogućile da uspješno izvršava dužnosti kardinala.

Pritom Nikola pripovjedni dio govora organizira po još jednom načelu: kronološki, od Riarijeva porijekla (MR 3) do smrtnog časa (MR 22). Govor tako ima dvije usporedne okosnice, životopis i vrline. Njihovo vješto isprepletanje vjerojatno je bio jedan od razloga uspjeha djela.

\section{Žanrovska pripadnost}

\section{Stil}

\section{Predaja teksta}

Nikolin govor za Pietra Riarija imao je sedam tiskanih izdanja i šest prijepisa.  Zbog teme djela i poznatih podataka o aktivnosti tiskara, može se pretpostaviti da je \textit{terminus post quem non} tiskanih izdanja, čak i tamo gdje nije naznačen datum, smrt pape Siksta IV, 12. kolovoza 1484.\footnote{Opširniji izvještaj o tiskarima vidi u Neven Jovanović, »Nadgrobni govor Nikole Modruškog za Pietra Riarija« \textit{Colloquia Maruliana} XXVII (2018), str.~123–141.}

Prikazat ćemo najprije tiskana izdanja, potom rukopise.

\subsection{Tiskana izdanja}

U kronologiji tiskanja skloni smo razlikovati dvije faze; prva bi bila neposredno povezana s Riarijevom smrću, dok drugu smještamo u godine oko 1482, kada se pojavljuje jedino datirano izdanje, ono padovansko Cerdonisovo. Važno je primijetiti da je tada Nikola Modruški već pokojan (umro je prije 29. svibnja 1480) te u ovoj fazi nije mogao imati udjela.

Sljedeći popis izdanja navodi kronološkim redom, prema datumima potvrđenima u kolofonu ili pretpostavljenima (uglate zagrade označavaju da je godina objavljivanja pretpostavljena), s mjestima i tiskarima. Izdanja označavamo siglom korištenom u kolaciji i kritičkom aparatu. Oznaka iza GW upućuje na zapis u računalnoj bazi podataka \textit{Gesamtkatalog der Wiegendrucke}.\footnote{Staatsbibliothek zu Berlin, Gesamtkatalog der Wiegendrucke / Inkunabelsammlung, www.gesamtkatalogderwiegendrucke.de/. Pristupljeno 29. studenoga 2019.}

\begin{description}[nolistsep,itemsep=3pt,font=\rmfamily]
\item[V\hphantom{e}] GW M26710, Rim [1474], In domo Antonii et Raphaelis de Vulterris (tiskara djelovala 1472–1474)
\item[V1] GW M26711, Rim [1474], In domo Antonii et Raphaelis de Vulterris\footnote{Mladen Bošnjak ustanovio je 1976, uspoređujući prijelom kolofona i posljednjeg lista primjerka koji se čuva u Trevisu s onim opisanim u katalogu British Library, da postoje dva različita izdanja: »Primjerak iz Trevisa također je proizašao iz oficine Antonii et Raphaelis Vulterris i ono je stvarno drugo izdanje, jer je očito, da tako skladan prelom djela mogao je nastati tek nakon već otisnutog izdanja, a koje je poslužilo kao predložak«. Mladen Bošnjak, »Dvije značajne hrvatske knjižice«, \textit{Hrvatska revija: Jubilarni zbornik 1951–1975}, Knjižnica Hrvatske revije, Barcelona, 1976, 590–598 i Mladen Bošnjak, »Zwei unbekannte Inkunabelausgaben«, \textit{Beiträge zur Inkunabelkunde} 3 (1983), 8, 91–93. Bošnjakova je identifikacija prihvaćena u \textit{Gesamtkatalog der Wiegendrucke}.}
\item[Ge] GW M26707, Rim [1474], Johannes Gensberg (djelovao 1473–1474)
\item[R\hphantom{o}] GW M26712, Rostock [1476], Fratres Domus Horti Viridis ad S.~Michaelem (tiskara djelovala 1476–1531)
\item[C\hphantom{d}] GW M26706, Padova 30.~kolovoza 1482, Matthaeus Cerdonis (djelovao 1482-1487)
\item[P\hphantom{d}] GW M26709, Rim [1482], Stephanus Plannck (djelovao 1479-1501)
\item[Gd] GW M26708, Rim [1482], Bartholomaeus Guldinbeck (djelovao 1475–1488)
\end{description}

Većini tiskara zajednička je orijentacija na mala, ponajviše prigodna izdanja; u taj se profil Nikolin govor savršeno uklapa. Znatan dio angažiranih poduzeća bio je kratka trajanja; tiskare de Vulterris, Gensbergova i Cerdonisova djelovale su pet godina ili manje. Antonio, jedan od braće Maffei, potvrđeno je bio povezan s moćnom obitelji Riario, baš kao i Nikola Modruški – možda se prvo izdanje govora za Riarija pojavilo upravo \textit{in domo de Vulterris}.\footnote{Braća Maffei radila su na papinskom dvoru kao pisari; bili su aktivni i kao sakupljači rukopisa i kao izdavači. Antonio Maffei sudjelovao je u takozvanoj uroti Pazzijevih. Na vezu tiskara i urotnika upozorila je Farenga, n. dj, 214, bilj. 95.}

\subsection{Rukopisi}

Svih šest danas poznatih rukopisnih prijepisa potječe iz posljednje četvrtine XV.~stoljeća. Četiri se čuvaju u Italiji – u Vatikanu, Rimu, Veneciji i Palermu – dok je po jedan u Njemačkoj (München), odnosno u Češkoj (Olomouc). Svi su prijepisi u humanističkim zbornicima (u kojima prevladavaju prozni sastavci). Svi su pisani humanistikom osim onog u Olomoucu, čije je pismo bastarda (iznimno raširena u Češkoj tijekom XIV. i XV. st). Nijedan prijepis nema bitnih naknadnih ispravaka ili marginalnih bilježaka. Također, prema onome što dosad znamo o ruci Nikole Modruškog,\footnote{Luka Špoljarić, »Ex libris Nicolai episcopi Modrussiensis: knjižnica Nikole Modruškoga«, \textit{Colloquia Maruliana} XXI, str.~25–63.} nijedan od prijepisa nije autograf i ne nosi tragove autorskih intervencija.

Donosimo osnovne podatke o prijepisima uz sigle kojima ćemo kodekse citirati u kritičkom aparatu.

\begin{description}[nolistsep,itemsep=3pt,font=\rmfamily]
\item[ve] Venecija, Biblioteca Marciana; Marc. Lat. cl. XIV, 180 (4667), XV. st, ff.~9r–19v.
\item[va] Vatikan, Bibliotheca apostolica Vaticana; Vat. lat. 8750; kraj XV. – početak XVI. st; ff.~205r–212v. Nekoliko folija ispred govora Modruškog (ff.~162r–172v) u ovom je kodeksu prepisan i nadgrobni govor Niccolòa Perottija (1429–1480) za kardinala Riarija.
\item[co] Rim, Accademia Nazionale dei Lincei, Biblioteca dell'Accademia dei Lincei e Corsiniana, fondo principale, cart. misc., XV. st, Corsin. 583 (45 C 18), ff. 117r–123r. Prepisivač je Tommaso Baldinotti (Pistoia, 1451–1511).\footnote{Identificiran u Armando Petrucci, »Alcuni codici Corsiniani di mano di Tommaso e Antonio Baldinotti«, \textit{Rendiconti dell' Accademia nazionale dei Lincei, classe di scienze morali, storiche e filologiche}, serie 8, XI (1956), str.~252–263.} Prije Nikolina govora, na ff.~113v–116v, prijepis je govora pred Sikstom IV. Nalda Naldija (1436 – nakon 1513), a nakon Nikolina slijedi govor Donata Accaiuolija (1429–1478) pred istim papom, ff.~123v–125v.
\item[pa] Palermo, Biblioteca centrale della Regione siciliana, I.B.6, oko 1474–1480, ff.~32r–54r. Datacija u katalogu određena je upravo prema Nikolinu djelu.
\item[m] München, Bayerische Staatsbibliothek, CLM 461, rukom Hartmanna Schedela (1440–1514), XV. st; ff.~129r–138r. Kodeks, u kojem najmlađi tekst potječe iz 1492, sadrži još nekoliko govora pred Sikstom IV.\ i govora povodom smrti drugih kardinala; radi se o djelima Bernarda Giustiniana (1408–1489), Lászla Vetésija (r.~oko 1460), Lodovica da Imola (djelovao 1462–1479) u smrt kardinala Pedra Ferriza (u.~1478), o anonimnom govoru u smrt kardinala Spoleta, te o govoru Gianantonija de San Giorgio (1439–1509) u smrt Ferrija de Clugny, kardinala i biskupa Tournaija (u.~1483).
\item[o] Olomouc, Vědecká knihovna, \textit{Textus uarii; Historia Bohemica,} sign. M I 159, oko 1476–1480, pisan u Litoměřicama;\footnote{Josef Truhlář, \textit{Počátky humanismu v Cechách}, V Praze: Nákladem České akademie císaře Františka Josefa pro vědy, slovesnost a umění, str.~507.} ff.~170r–174v.
\end{description}

\subsection{Kolacija svjedoka i stemma codicum}

Kolacija tiskanih i rukopisnih svjedoka predaje otkriva podjelu u dvije glavne obitelji. Izvan tih dviju skupina ostaju četiri izvora za koja nismo spremni tvrditi da pripadaju samostalnoj obitelji. 

Obitelj 1 čine četiri svjedoka. To su izdanja Johannesa Gensberga (\textit{Ge}, Rim, nakon 18. siječnja 1474) i Mateja Cerdonisa (\textit{C}, Padova, 30. kolovoza 1482) te rukopisi \textit{va} i \textit{o}. Oba su rukopisa, čini se, prepisana iz Gensbergova izdanja. 

I u obitelji 2 četiri su svjedoka: dva izdanja in domo de Vulterris (\textit{V, V1;} Rim, nakon 18. siječnja 1474), te njima bliska izdanja Stefana Planncka i Bartolomeja Guldinbecka (\textit{P, Gd;} oba u Rimu, oko 1482). Rukopis \textit{m} zasigurno je prijepis Plannckova izdanja. 

Nešto bliže potonjoj obitelji – ali ne bismo se usudili tvrditi da se radi o izravnoj vezi – stoji izdanje Fratres Domus horti viridis ad S. Michaelem (\textit{R}, Rostock, 1474. ili kasnije) te rukopisi \textit{ve, pa} i \textit{co; pa} je bliži \textit{ve} nego \textit{co.}

Ovi su međusobni odnosi shematski prikazani kao \textit{stemma codicum} na priloženom dijagramu (slika 1).

U nastavku navodimo najvažnije tekstualne inačice na osnovi kojih smo predložili ovakvu podjelu svjedoka. Brojevi odlomaka slijede podjelu u ovom kritičkom izdanju.\footnote{Opširniji popis različitih čitanja vidi u aparatu, te u Neven Jovanović, »Rukopisi, kolacija svjedoka predaje i paratekstovi nadgrobnog govora Nikole Modruškog za kardinala Pietra Riarija (1474)« (rad u postupku ocjenjivanja).}

\subsection{Obitelj 1 (V Gd P m)}

\begin{enumerate}[label=\alph*)]
\item MR 4 \textlatin{inter cenandum de uariis disciplinarum studiis frequenter disserere consueuerat adeo acute adeoque prompte ac subtiliter de quaestione proposita ut eum putares die noctuque nulli adeo alii rei quam euoluendis theologorum philosophorumque libris uacare.} – sva četiri donose sintaktički nelogično »putare« nasuprot »putares« \textit{(R va ve pa co),} odnosno »putaret« \textit{(Ge C o)}
\item MR 4 \textlatin{ex perceptis semel principiis difficillima quaeuis uel philosophiae uel theologiae problemata summa cum omnium admiratione absoluebat} – sva četiri ispuštaju »uel theologiae«
\item MR 11 \textlatin{seque ab inuidis atque malignis impie ac flagitiose eius criminis insimulari persancte iurabat.} – umjesto »inuidis« \textit{V P m} imaju neobično »infidiis«, \textit{Gd} nešto prihvatljivije »inuidiis« (svi ostali »inuidis«, kao paralelu pridjevu »malignis«)
\item MR 17 \textlatin{Et licet nonnumquam, domesticorum amicorumque euictus precibus, aut litteris aut nuntiis multos iudicibus commendaret, id tamen citra cuiusque iniuriam fieri uolebat.} – umjesto »citra«, \textit{V Gd P} imaju »circa«; u \textit{m} je zabilježena i inačica: »circa al. citra« (svi ostali imaju »citra«)
\item MR 19 \textlatin{testatur Taruisii maior basilica non paruis ditata uectigalibus} – sva četiri donose neobično »Taruixit« umjesto »Taruisii« \textit{(co} ima »Taruixij«); nadalje, \textit{V Gd P} imaju »non parua« (što je zalihosno), ostali čuvaju litotu čitanjem »non paruis«
\item MR 20 \textlatin{Delicta, quae uel aetatis uel fortunae uitio pro fragilitate humana contraxerat, pia confessione saepius diligentiusque purgauit et munitus caelesti uiatico, quod summa cum deuotione acceperat, diuinam uoluntatem accinctus praestolabatur.} – nasuprot svim ostalima, \textit{V Gd P} imaju nezgodno čitanje »minutus« (kao da je umirući »umanjen za nebesku popudbinu«, tj.\ da je ostao bez nje)
\item MR 20 \textlatin{non fortunam accusauit nec se in medio iuuentutis flore ex tanto imperio et ex talibus opibus subtrahi uel leuiter indoluit} – sva četiri imaju »operibus«, nasuprot »opibus« kod ostalih; »operibus« je ponešto laskavije po Riarija, kojeg smrt u tom slučaju ne bi odvlačila od bogatstva, nego od (započetih) pothvata
\item MR 20 \textlatin{»Sentio,« inquit, »filii fratresque mei, manum Domini super me aggrauari; uolens lubensque eius praesto sum uoluntati, eo quidem libentior quo me et famae et gloriae meae satis uixisse scio.«} – sva četiri (također i \textit{pa)} imaju »nolens«, a ne »uolens«, što dramatično izvrće značenje Riarijevih riječi
\end{enumerate}

\subsection{Veza Gd P m}

\begin{enumerate}[label=\alph*)]
\item MR 2 \textlatin{Extinctus iacet optimarum artium deditissimus amator} – Gd i P imaju gramatički neumjesno »Extinctis«, svi ostali svjedoci »Extinctus«
\item MR 2 \textlatin{Nolite igitur, nolite expectare, praestantissimi patres} – u »Nolite igitur, nolite« ova su tri svjedoka ispustila drugo »nolite«, poništavajući tako dijakopu
\item MR 4 \textlatin{Clarescebat autem iam tunc nomen religiosissimi doctissimique uiri, magistri Francisci, conciuis et auunculi sui} – ova tri svjedoka imaju sintaktički neobično »Clarescat« (ostali »Clarescebat«, osim \textit{C} »Clare sciebat«)
\item MR 4 \textlatin{Quem ubi conspexisset Franciscus iam religionis ueste indutum} – umjesto »ueste«, \textit{Gd} i \textit{P} imaju »iuste« (svi ostali »ueste«); \textit{m} je očito zapazio nelogičnost pri prijepisu
\item MR 5 \textlatin{Quibus uarie sollicitatus coepit auunculum suum hortari, coepit importunius compellere Romam peteret} – umjesto »compellere« sva tri imaju »complere« (uz potonji bismo glagol, upotrijebljen u prenesenom smislu, očekivali oznaku \textit{čime} je nećak »prekrivao« ili »zasipavao« ujaka)
\item MR 10 \textlatin{Accipiebat praeterea munera, non auaritiae, sed honoris comparandaeque beniuolentiae gratia} – umjesto »comparandaeque« \textit{Gd} ima »comperandique«, \textit{P m} »comparandique«; uz gerund bi trebalo biti »comparandique beniuolentiam«, no tada se ne bi poštovala težnja biranog latinskog stila da gerund zamijeni gerundivom
\item MR 11 \textlatin{Vidi illum quodam uesperi non sine graui stomacho lacrimis suffusum oculos et cum multa indignatione Deum optimum maximumque testem citare} – umjesto »et (cum multa\dots)« sva tri imaju gramatički neumjesno »te«
\item MR 14 \textlatin{Omnium saluti die noctuque inseruiebat, et tamen a nonnullis negligentiae accusabatur} – umjesto »accusabatur« \textit{Gd} i \textit{P} imaju sintaktički nelogično »accusabantur«, kao da su »negligentiae« subjekt
\item MR 16 \textlatin{excepto hoc piissimo fratre comite Hieronymo; quem quoniam ab inclito duce Mediolani connubio filiae dignatum cernebat} – umjesto »connubio« sva tri imaju »connubia«, kao da je objekt predikata »cernebat«
\item MR 19 \textlatin{Proinde non cessabat ecclesias suae curae commendatas collapsas erigere, exornare deformes; praedia occupata uendicare, bona priorum rectorum distracta negligentia propriis pecuniis recuperare; uestimenta, libros, uasa sacra et caetera ad splendorem diuini cultus spectantia maximis sumptibus coemere} – umjesto »uasa« \textit{Gd} i \textit{P} imaju gramatički neodrživo »uaso«
\item MR 20 \textlatin{ut uos horter atque obtester ne huius mundi illecebris atque lenociniis animum uestrum inducatis neue in luxu ac inanibus eius diuitiis spem ullam ponatis} – umjesto »inanibus« (\textit{va} »in anibus«) \textit{Gd} i \textit{P} imaju »manibus«, riječ koja je sama za sebe smislena, ali u rečenici neumjesna
\end{enumerate}
\subsection{Tijesna veza P m}

\begin{enumerate}[label=\alph*)]
\item MR 3 \textlatin{quoniam et honestissimis nobilissimisque ciuitatis suae parentibus est ortus et celeberrimo uetustoque Ligurum oppido Saona} – oba svjedoka imaju »Soana«, ostali »Saona«
\item MR 4 \textlatin{uersus complures multosque grammaticae textus, quos olim puer edidicerat} – oba svjedoka imaju »edicerat«, ostali »edidicerat«
\item MR 5 \textlatin{Cerno hic nonnullos praelatos et ex aliis ordinibus uiros praestantes a quibus magna cum attestatione audiui} – oba svjedoka imaju gramatički nemotivirano »praestante«, ostali »praestantes«
\item MR 5 \textlatin{dicens in manu solius omnipotentis Dei esse omnia regna terrarum atque illa, quibus ipse uoluerit, tradi} – oba imaju sintaktički neopravdano »uoluerint«, ostali »uoluerit«
\item MR 6 \textlatin{simulque cum eis magnanimitatem, clementiam, munificentiam, et ceteras, quas prius commemorauimus, imperantium uirtutes} – \textit{V P m} imaju neobično »prius quas prius«, ostali samo »quas prius«
\item MR 8 \textlatin{Mali quippe et iniqui hominis esse dicebat meliorem se foris ostendere quam gerere domi} – \textit{V P m} imaju (nelogično) »fortis«, nasuprot »foris« kod ostalih (osim »fori« kod \textit{Ge C)}
\item MR 8 \textlatin{liberius argueretur quod nimia indulgentia et largitate domesticos faceret insolentiores} – \textit{P m} imaju negramatično »indulgenti et a«, ostali »indulgentia et«
\item MR 8 \textlatin{placida uoce respondit} – \textit{P m} imaju »placenda«, ostali »placida«
\item MR 17 \textlatin{Quattuordecim enim milia ducatorum accepit et opulentissimum oppidum Bosti, ex quo et aliis a duce adiectis praediis plus quam quinque milia ducatorum quotannis capere poterit} – umjesto »et opulentissimum« \textit{P m} imaju negramatično »hec opulentissimum«, \textit{V} »ec opulentissimum« (svi ostali »et opulentissimum«)
\item MR 17 \textlatin{»Illud« inquit »ut iustitiam mearum precum gratia minime uioles, nec secus feceris etiam si te germanus meus Hieronymus\dots«} – \textit{P m} imaju uiolet (kao da se Riario ne obraća izravno upravitelju grada Rima, bez obzira na »secus feceris« koje neposredno slijedi), ostali »uioles«
\item MR 20 \textlatin{Iamque uicinus morti domesticos ac familiares accersiri iubet, quibus praesto existentibus in nullos prorupit fletus, nullis mundanarum cupiditatum desideriis ingemuit} – dok ostali imaju »nullis«, u \textit{P m} stoji vrlo nezgodno »multis« (što surečenici daje upravo suprotan smisao)
\item MR 21 \textlatin{Minus quippe meae iuuentutis potens fui et nonnunquam partim oculos, partim aures uestras in multis offendi. Sed eorum tanto facilius me a Domino misericordiam consecuturum confido quanto uos modestius uiuentes pro me Dominum deprecabimini. Ego quoque, si quis mortuis erit sensus, idem pro uobis me spondeo facturum.} – \textit{P m}, opet sadržajno neugodno, ispuštaju čitavu rečenicu »Sed\dots\ deprecabimini«, što dramatično iskrivljuje smisao odlomka (»i vaše oči, i uši umnogome sam povrijedio. A i ja ću, tvrdo obećajem, i za vas učiniti isto\dots«)
\item MR 21 \textlatin{Viuite mei memores et, quam caduca sit huius mundi felicitas, uel meo exemplo discite} – gdje ostali imaju »Viuite«, \textit{P m} imaju »Uite« (što se možda može shvatiti kao genitiv imenice \textit{uita,} »pamteći moj život«)
\end{enumerate}
\subsection{Obitelj 2 (Ge C va o)}
\begin{enumerate}[label=\alph*)]
\item MR 3 \textlatin{eas laudes, quas uel a parentibus uel a patria ipsius colligere poteram, hoc loco praetermittendas putaui; non quod illas aut obscuras aut tenues fore duxerim} – \textit{Ge C va o} imaju »duxerim«, nasuprot »dixerim« kod ostalih (obje su varijante jezično i smisleno prihvatljive)
\item MR 6 \textlatin{Assumpsit enim cum sublimi magistratu sublimes animos et spiritus tanti imperii maiestate dignos simulque cum eis magnanimitatem} – ova četiri svjedoka imaju zalihosno »et simulque« nasuprot »simulque« kod ostalih
\item MR 19 \textlatin{Haec quoque sacra apostolorum aedes beneficentiam eius testari potuisset si tantum quattuor mensibus superstitem uidisset.} – sva četiri imaju »et si (tantum quattuor\dots)«, ostali si; et si bi rečenici dalo neželjen dopusni prizvuk (»čak i da ga je bazilika vidjela kako je još četiri mjeseca poživio«)
\item MR 22 \textlatin{Ad hanc uocem illa dilecta Deo anima ueluti certo accepto signo ad Dominum suum confestim euolauit.} – u opisu Riarijeve smrti, sva četiri svjedoka ispuštaju čitavu ovu rečenicu, čime se (možda iz razloga dobrog ukusa) ublažava inzistiranje na svetosti kardinalovih posljednjih trenutaka; ovo smatramo ključnom razlikom za podjelu na grane u predaji
\end{enumerate}
\subsection{Veza Ge va o}
\begin{enumerate}[label=\alph*)]
\item MR 2 \textlatin{Extinctus iacet optimarum artium deditissimus amator, interiit omnium studiosorum praecipuus fautor} – umjesto »studiosorum« ova tri svjedoka imaju »studiorum«, i time unekoliko mijenjaju značenje (u toj verziji Riario nije zaštitnik znanstvenika, nego znanosti)
\item MR 5 \textlatin{ut discerent uniuersi ueram certamque esse illam Nabuchdenosoris confessionem in quam et regno et sensui restitutus supplex prorupit dicens in manu solius omnipotentis Dei esse omnia regna terrarum} – \textit{Ge} i \textit{o} imaju gramatički neodrživo »eā« (skraćenicu za »eam«), \textit{va} piše »ea«, kao da se odnosi na »regna«; ostali svjedoci imaju »esse«
\end{enumerate}
\subsection{Poseban položaj R ve pa co}
\begin{enumerate}[label=\alph*)]
\item MR 1 \textlatin{Cum in omni funebri celebratione duo praecipue dicendi genera a maioribus nostris usurpari soleant} – samo ova četiri svjedoka ispuštaju »omni«
\item MR 23 \textlatin{nec sibi nec gloriae suae parum uixit qui, quaecumque unius hominis fortuna capere potuit, abunde consecutus est. Nobis forsan amplius uiuere poterat, et nimirum magno et ornamento et utilitati.} – samo ova četiri iz druge rečenice ispuštaju »uiuere«, kao da se podrazumijeva prema prvoj
\end{enumerate}
\subsection{Veza ve pa}
\begin{enumerate}[label=\alph*)]
\item MR 4 \textlatin{ut eum putares die noctuque nulli adeo alii rei quam euoluendis theologorum philosophorumque libris uacare} – oba svjedoka ispuštaju »adeo« i time poništavaju hiperbat
\end{enumerate}

\subsection{Paratekstovi u mletačkom kodeksu}

Samo mletački rukopis \textit{(ve)} na posljednjoj stranici prijepisa Nikolina govora (f.~19v), odmah nakon završetka glavnog teksta, donosi četiri pohvalna epigrama od po dva stiha (odnosno, po jedan elegijski distih). Pjesme su očito bile predviđene kao paratekstovi tiskanog izdanja. Prvi epigram hvali Nikolin govor – pohvala potvrđuje i kontroverznost Riarijeve reputacije i planiranu apologetsku funkciju govora – dok su ostala tri varijante epitafa za pokojnog kardinala. Sve pjesme donosimo ovdje, uz naš prijevod.

\begin{quote}
In laudem libelli\\
Ęloquio uires quantę sint, aspice, lector\\
	Quem prius odisti, protinus hunc peramas.

U slavu knjižice\\
Čitaoče, uvjeri se kakva je u riječima snaga:\\
	kog si nekoć mrzio, sad si odmah zavolio.

\bigskip

Epithaphion(!)\\
Nemo magis docuit perituri temnere sęcli\\
	Diuitias, fastum, luxuriemque simul.

Epitaf\\
Nitko nas nije poučio bolje prezirati prolaznog svijeta\\
	raskoš, bogatstvo i sva blaga.

\bigskip

Aliud.\\
Quam celeris fugiat ruituri gloria mundi\\
	Me speculum cernas quisque uiator ades.

Još jedan.\\
Kako brzo nestaje slava krhkog svijeta,\\
	vidi u meni, ogledalu svojem, putniče, ma tko bio.

\bigskip

Aliud.\\
Sorte humili natum qui me cognouerit ante\\
	Fortunę uarios rideat ille iocos.

Još jedan.\\
Tko je znao kako skromnog sam roda, nek se smije\\
	šalama raznovrsnim što ih zbija hir sudbine.

\end{quote}

Ne znamo ni autora ni vrijeme nastanka ovih stihova. Pjesme variraju motivima i formom, kako i dolikuje nadgrobnome ciklusu. Stihovi su složeni metrički besprijekorno. Odabir riječi i sintaksa ne odstupaju od uzusa rimske književnosti. Nepoznati autor ne »reciklira« dijelove antičkih stihova.\footnote{Usporedba s korpusom sačuvane rimske poezije tek u trećoj pjesmi pronalazi paralele sa stihom Lukrecijeva epa \textit{O prirodi} i Ovidijevih \textit{Fasta}: \textlatin{quisque uiator ades: Ovid. F. 3, 351–352: »At certe credemur« ait »si uerba sequetur / Exitus: en audi crastina, quisquis ades,«} (govori Numa Pompilije). – \textlatin{Quam celeris: Lucret. 4, 210–213: Quam celeri motu rerum simulacra ferantur, / Quod simul ac primum sub diu splendor aquai / Ponitur, extemplo caelo stellante serena / Sidera respondent in aqua radiantia mundi} (o brzini kojom se odraz zvijezda pojavljuje na vodi).} Nešto su učestalije paralele s latinskom poezijom talijanskih suvremenika Nikole Modruškog – no, i ovdje su odjeci daleki, više u sličnostima nego u identičnosti.\footnote{Višekratno nailazimo na paralele s poezijom Landinove zbirke \textit{Xandra} (dovršena 1460) i Pontanova \textit{Eridana} (zbirka započeta 1483). Evo popisa paralela: \textlatin{aspice, lector: \textit{Anthologia Latina} 855d, 1–2: Volue tuos oculos: metuendum hunc aspice, lector, / Armorum bellique ducem} (natpis pod likom Julija Cezara; autor ovog epigrama je Francesco da Fiano, oko 1350–1421). – \textlatin{Nemo magis docuit: Cristoforo Landino (Firenca, 1424–1498), \textit{Xandra} 3, 17 \textit{(Ad Petrum Medicem de laudibus Poggi,} 1456–1458), 37–38: Nemo magis dubiis potuit cognoscere rebus / Utile nec docto promptius ore loqui. – Diuitias, fastum, luxuriemque simul: Bonvesin da la Riva (Milano, oko 1240 – oko 1315), \textit{Vita scolastica,} 187–188: Privat, sternit opes, viciat, scelus omne ministrat, / Furta docet, predas luxuriamque simul; isto, 773–774: A viciis caveat, virtutibus hereat, absint / Fastus avaricie luxurieque fimus.} (Bonvesinova je poučna pjesma bila vrlo popularna u Quattrocentu, postoje i brojna tiskana izdanja). – \textlatin{Me speculum cernas quisque uiator ades: Giovanni Gioviano Pontano (Cerreto di Spoleto 1429 – Napulj 1503), \textit{Eridanus} 2, 2 \textit{(Puerum alloquitur faculam nocturnam praeferentem,} oko 1490), 11–12: Quisquis ades Stellamque vides, mea pectora cerne / In speculo; speculum pectoris illa mei est – Sorte humili natum: Cristoforo Landino, \textit{Xandra} 3, 6 \textit{(Ad Paulum ne timeat bellum Aragonense,} 1452), 69–70: Per quos ah casus, per quanta pericula cernes / Ex humili hunc summum iam tetigisse gradum. – Fortunę uarios rideat ille iocos: Albertino Mussato (Padova 1261 – Chioggia 1329), \textit{De obsidione} 1 \textit{(De obsidione domini Canis Grandis de Verona circa menia Paduane civitatis),} 730–731: Victores victique pari non mente rotatum / Fortune videre iocum fatique profundi; \textit{De obsidione} 3 \textit{(De conflictu domini Canis apud Paduam),} 218–219: O nunquam stabilis, vario diversa rotatu, / Rebus in humanis quantum, Fortuna, iocaris! Petrarca (Arezzo 1304 – Padova 1374), \textit{Africa} 7, 419–421: Scimus: et hinc maior nostris speratur ab armis / Gloria. Nec regnum Fortune ignoro iocantis / Rebus in humanis. – Ugolino Verrino (Firenca, 1438–1516), \textit{Flametta} (1463) 1, 4 \textit{(Ad Flamettam),} 35–36: O si mixta viris spartanae more palestrae / Tractaret varios tusca puella iocos} (Verrino je Landinov učenik, \textit{Flametta} slijedi uzor zbirke Xandra). – \textlatin{Pontano, \textit{Eridanus} 2, 31 \textit{(Ad Marcum Antonium Sabellicum scriptorem historiarum,} nakon 1487), 25–26 Stant et opes animi validae; ridemus iniquas / Fortunae insidias instabilisque vices.}}

\section{Načela ovog izdanja}








\end{document}
\section*{1.\thinspace LIBRI IMPRESSI}


\section*{2.\thinspace CODICES}

\bigskip

\begin{description}[noitemsep,itemsep=3pt,labelsep=5pt,font=\rmfamily] 
\item[ve] -- Codex Venetus, \emph{Marc.~Lat.\ cl.~XIV, 180 (4667),} saec.~XV, ff.~9r–19v.
\item[va] -- Codex Vaticanus: Bibliotheca apostolica Vaticana \emph{Vat.~lat.\ 8750;} saec.\ XV exeunte – XVI ineunte; ff.~205r–212v; e \textit{Ge} descriptus esse uidetur.
\item[pa] -- Codex Panormitanus: Panormi, Biblioteca centrale della Regione siciliana, \emph{I.B.6,} c.~1474–1480; ff.~32r–54r.
\item[co] -- Codex Corsinianus: cart.~misc., saec.~XV; Romae, Accademia Nazionale dei Lincei, Biblioteca dell'Accademia dei Lincei e Corsiniana, fondo principale, \emph{Corsin. 583 (45 C 18),} ff.~117r–123r.
\item[m\hphantom{i}] -- Codex Monacensis: Bibliotheca Bauarica \emph{CLM 461}, ab Hartmanno Schedelio scriptus, saec.~XV; ff.~129r–138r; descriptio \textit{P} uidetur esse.
\item[o\hphantom{o}] -- Codex Olomoucensis, sub fine saec. XV; Olomoucii, Vědecká knihovna; Textus uarii; Historia Bohemica \emph{sign. M I 159}, ff.~170r–174v; e \textit{Ge} descriptus esse uidetur.
\end{description}


\clearpage
\thispagestyle{empty}
\hfill
\clearpage
