%-*-coding:utf-8;-*-
%&LaTeX
\documentclass[a5paper,twoside]{article}
\usepackage{polyglossia}
\setmainlanguage{croatian}
\setotherlanguage{latin}

\defaultfontfeatures{Ligatures=TeX}

\usepackage{indentfirst,titling,enumitem}

% prilagodba dimenzija stranice
\usepackage[a5paper,pdftex]{geometry}%a4paper,pdftex
%\usepackage[a4paper,pdftex]{geometry}
\geometry{scale=1,paperwidth=161mm,paperheight=230mm,
bindingoffset=0in,top=27mm,inner=21mm,outer=26mm,width=117mm,height=180mm}%
\parindent=7mm

%\usepackage[small,center,rm]{titlesec}
%\titlespacing{\section}{0pt}{11mm}{7mm}

\usepackage[medium,center,rm]{titlesec}
\titlelabel{\thesection.\enspace}
\titlelabel{\thesubsection.\enspace}
% \chapter, \subsection...: no additional code
\titleformat{\section}
  {\normalfont\large\bfseries\uppercase}
  {\thesection.\enspace}{0em}{}
\titlespacing{\section}
  {7mm}{3.5ex plus .1ex minus .2ex}{1.5ex minus .1ex}
\titleformat{\subsection}
{\normalfont\large\bfseries}
{\thesubsection.\enspace}{0em}{}
\titlespacing{\subsection}
  {7mm}{3.5ex plus .1ex minus .2ex}{1.5ex minus .1ex}


% titling
\pretitle{\begin{center}\vskip 5mm}
\posttitle{\par\end{center}}
\preauthor{}
\postauthor{}
\predate{}
\postdate{}

% slova
\setmainfont{Times New Roman}
%\setsansfont{Old Standard TT}

% makro za regum gesta
\newcommand{\kratica}{\emph{Regum gesta}}%

% reference bold
\newcommand{\refb}[1]{\textbf{\ref{#1}}}

\renewcommand{\labelenumi}{\Alph{enumi}}

%tekuće glave
%tekuće glave
\usepackage{fancyhdr}
\pagestyle{fancy}
\fancyhead{} % clearall
\fancyhead[RO,LE]{\thepage}
\fancyhead[EC]{\MakeUppercase{Nikola Modruški: Djela u službi pape Siksta IV.}}
\fancyhead[OC]{\MakeUppercase{Govor za Pietra Riarija (1474)}}
\fancyfoot{}
\renewcommand{\headrulewidth}{0.4pt}

\makeatletter
\renewcommand{\@makefnmark}{\mbox{%
\textsuperscript{\normalfont\@thefnmark\ }}}
\makeatother

\renewcommand\footnotesize{\fontsize{10}{10.5} \selectfont}
\renewcommand\tiny{\fontsize{9}{9.5} \selectfont}
\renewcommand\large{\fontsize{11}{11.5} \selectfont}
\renewcommand\Large{\fontsize{12.5}{13} \selectfont}

% fusnote i crta
\renewcommand{\footnoterule}{\vspace*{-7pt}
    \noindent\rule{2in}{0.4pt}\vspace*{6.6pt}}

\setlength{\headheight}{12.6pt}

% makro za naziv programa
\newcommand{\makeup}[1]{\MakeUppercase{\emph{#1}}}


% prilagodba quote
% prilagođavamo citat propozicijama
\renewenvironment{quote}
               {\list{}{\rightmargin 0mm
                \leftmargin 7mm
                \itemindent 0em}%
                \item\relax}
               {\endlist}

\newenvironment{opisi}{%
\let\olditem\item% 
\renewcommand\item[2][]{\olditem ##1 ##2 \itemsep 3pt}%
\begin{description}}{\end{description}%
}


%siročići i udovice
% \widowpenalty=10000
% \clubpenalty=10000

\begin{document}
%%tth:\begin{html}<META HTTP-EQUIV="CONTENT-TYPE" CONTENT="text/html; charset=utf-8">\end{html}

\hyphenation{his-to-ri-o-graf-ske po-li-tič-ko po-li-tič-ko-his-to-ri-o-graf-ske}
\hyphenation{quae-sti-o-nes quin-qua-gin-ta quo-que DXXXVIII Zuo-ni-me-ro}
\hyphenation{quad-ra-ta quo-mo-do Schwand-tne-ro-vi cro-a-tiae I-su-krs-ta}
\hyphenation{Pas-qua-li-go XVIII U-di-sla-va in-tel-le-xis-set kra-ljev-stvo}
\hyphenation{in-ua-se-runt quod pes-si-mo pe-dan-tno-an-ti-kvar-ski pas-cha-les}
\hyphenation{Pa-lu-šom MDCCXLVIII}

% No extra spacing after periods
\frenchspacing

% Font sizing
\fontsize{11}{13.2}
\selectfont

% veća čitljivost
\linespread{1.1}

%uvlaka paragrafa
\setlength{\parindent}{7mm}

% počinje na s. 5
\setcounter{page}{169}

\title{\Large{\MakeUppercase{Govor za Pietra Riarija (1474)}}}

\date{}%Datum ove verzije: \today}

\maketitle 

%\texttt{Završna verzija, uredniku na čitanje.}
\thispagestyle{empty}
% [ova stranica namjerno prazna]

\section{Uvodne napomene}

Latinski nadgrobni govor modruškog biskupa Nikole za kardinala Pietra Riarija (preminulog 5. siječnja, sahranjenog 18. siječnja 1474. u Rimu) igrom je slučaja prva poznata nam tiskana knjiga nekog hrvatskog autora, svojevrstan latinski pandan devet godina kasnije tiskanog glagoljskog \textit{Misala po zakonu Rimskoga dvora.}

Autograf djela nije sačuvan. Priređujući kritičko izdanje i prvi hrvatski prijevod Nikolina govora,\footnote{Znanstveno je izdanje, kako ga je priredio autor ove studije, dostupno od 2009.\ u digitalnoj zbirci \textit{\textlatin{Croatiae auctores Latini}}; prema tom izdanju govor je nedavno preveden na engleski: Brendan Cook i Jennifer Mara DeSilva, »Princely Ambiguity: A Translation of Nikolaus of Modruš’ Funeral Oration for Cardinal Pietro Riario: Oratio in funere Petri Cardinalis Sancti Sixti (1474)«, \textit{Royal Studies Journal,} 5 (2018), 2, str.~92–128.} željeli smo uspostaviti najbolji mogući tekst, omogućiti uvid u tekstualne inačice trinaest svjedoka predaje, te djelo učiniti jezično i stvarno što razumljivijim današnjim čitaocima.

Boljem razumijevanju namijenjen je i ovaj uvod. Predstavit ćemo najprije povijesni kontekst u kojem su se pojavljivala izdanja govora za Riarija, potom tematsku strukturu, žanrovsku pripadnost i stilska obilježja samog djela, da bismo na koncu prikazali svjedoke predaje teksta, njihov međusobni odnos, i načela ovog izdanja. 

\section{Povijesni kontekst}

\section{Tematska i retorička struktura govora}

Nikolin govor za Riarija ima jasnu i jednostavnu strukturu (\textit{\textlatin{dispositio}} antičke retorike): uvod, glavni dio i završetak. Strukturu ovdje dodatno tematski raščlanjujemo navodeći brojeve odlomaka u kritičkom izdanju (uz siglu MR) i dodatno citirajući početke i krajeve pojedinih cjelina. Napomene u zagradama standardni su retorički termini za pojedini dio govora ili pojedine ključne riječi.

\begin{description}[nolistsep,itemsep=3pt,font=\rmfamily]
\item[1. MR 1–2 Uvod \textlatin{(exordium, prooimion)}] – \textlatin{Cum in omni funebri celebratione\dots\ quam ingratitudinis notam subire uerebor.}
\item[2. MR 3–22 Pripovjedni dio (narratio)] Riarijev život, vrline, smrt: \textlatin{Dicturus igitur de laudibus reuerendissimi domini\dots\ ad dominum suum confestim euolauit.}
\begin{description}[nolistsep,itemsep=3pt,font=\rmfamily]
\item[2.1. MR 3] Domovina i rod: \textlatin{eas laudes, quas uel a parentibus uel a patria\dots\ omnibus futuris seculis non desinent celebrari.}
\item[2.2. MR 3] Vrline: \textlatin{Quae quidem tametsi satis grandis eius gloria sit\dots\ quae quales in eo fuerint, breuiter explicare conabor.}
\item[2.3. MR 4–5] Dječaštvo: \textlatin{Qua igitur pietate primum erga deum\dots\ aut dicere illi quare sic fecisti.}
\begin{description}[nolistsep,itemsep=3pt,font=\rmfamily]
\item[2.3.1. MR 4] Dječaštvo – stupanje u franjevački red i pobožnost \textlatin{(pietas): Qua igitur pietate primum erga deum\dots\ seruatisque pro more religionis rite caerimoniis uestem Christi induit.}
\item[2.3.2. MR 4] Dječaštvo – školovanje i mudrost \textlatin{(prudentia): Qua assumpta ita omnia tyrocinii rudimenta libens promptusque\dots\ ita memoriter recitabat ut ea illum heri aut nudiustertius memoriae mandasse existimares.}
\item[2.3.3. MR 5] Dječaštvo – dolazak u Rim i proročka moć \textlatin{(divinitas): Perfectis igitur quam celerrime omnium bonarum artium studiis\dots\ aut dicere illi quare sic fecisti.}
\end{description}
\item[2.4. MR 6–11] Kardinalska čast – dužnosti kardinala \textlatin{(officia): Petrus igitur per hunc modum in principatu constitutus\dots\ in uiros praestantes ac bene meritos officiosum esse.}
\begin{description}[nolistsep,itemsep=3pt,font=\rmfamily]
\item[2.4.1. MR 6] Dužnosti — darežljivost \textlatin{(munificentia): Turpe enim et indecorum merito ducebat\dots\ culpam magis dolens quam damnum.}
\item[2.4.2. MR 7] Dužnosti – ustrajnost \textlatin{(perseverantia): A suscepto semel negotio nullo metu\dots\ magnanimitatis eius testimonia.}
\item[2.4.3. MR 7] Dužnosti – milostivost \textlatin{(clementia): Sed illud mea sententia uincit uniuersa\dots\ ut inimicis suis benefacere gauderet.}
\item[2.4.4. MR 8] Dužnosti – iskrenost \textlatin{(sinceritas): Nihil in se fictum, nihil subdolum\dots\ sed integritate uiros superare.}
\item[2.4.5. MR 8–9] Dužnosti – kardinalski dvor \textlatin{(familia)} i darežljivost \textlatin{(munificentia et liberalitas): Sane munificentiae liberalitatisque eius\dots\ bonorum morum gratia diuersari licet.}
\item[2.4.6. MR 10] Dužnosti – primanje darova: \textlatin{Accipiebat praeterea munera, non auaritiae\dots\ multa insuper dona tulerunt.}
\item[2.4.7. MR 11] Dužnosti – optužbe za simoniju: \textlatin{Vidi illum quodam uesperi non sine graui stomacho\dots\ in uiros praestantes ac bene meritos officiosum esse.}
\end{description}
\item[2.5. MR 12] Patetični ekskurs protiv zavidnika: \textlatin{Vbi nunc sunt rubiginosa illa maliuolorum pectora\dots\ tota die concinnabant dolos.}
\item[2.6. MR 12–13] Povratak na glavnu temu (hipostrofa): \textlatin{Sed haec illi uiderunt\dots\ magna ex parte potuit esse manifesta.}
\item[2.7. MR 13–14] Mudrost (prudentia) na dužnosti: \textlatin{Quamobrem tantum eam partem attingam\dots\ ut socium crederes, non dominum.}
\begin{description}[nolistsep,itemsep=3pt,font=\rmfamily]
\item[2.7.1. MR 13–14] Mudrost u politici (diplomatska misiji u Italiji, prihvaćanje »uresa«, zalaganje za mir u Italiji): \textlatin{Ostendit id nouissima haec ipsius legatio\dots\ ni eum nobis haec dira atque crudelis mors tam repente praeripuisset.}
\item[2.7.2. MR 14] Zauzetost za opće dobro \textlatin{(omnium salus): Vincebat ingenio humana consilia\dots\ ut socium crederes, non dominum.}
\end{description}
\item[2.8. MR 15–16] Umjerenost \textlatin{(moderatio): Tanta uero animi moderatione erat\dots\ urbem illam magis ecclesiae quam fratris gratia uoluit uendicare.}
\item[2.9. MR 17] Pravičnost \textlatin{(iustitia): Porro iustitiam ipsius ex illo spectare licet\dots\ pro sua benignitate dissoluat.}
\begin{description}[nolistsep,itemsep=3pt,font=\rmfamily]
\item[2.9.1. MR 17] Pravičnost pri povratu grada Imole: \textlatin{Porro iustitiam ipsius ex illo spectare licet\dots\ neque Imolam nisi eo uolente redimere uoluit.}
\item[2.9.2. MR 17] Pravičnost u ophođenju s dužnosnicima: \textlatin{Magistratus hortabatur ius suum absque ullo respectu\dots\ non nisi iustum honestumque fieri permaxime uelle.}
\item[2.9.3. MR 17] Pravičnost u izdavanju isprava: \textlatin{Hinc et rescripta non nisi sanctissima\dots\ apud notarios exstare iusserat.}
\item[2.9.4. MR 17] Pravičnost u sastavljanju oporuke: \textlatin{Huius praeclarissimae uirtutis nec moriens\dots\ pro sua benignitate dissoluat.}
\end{description}
\item[2.10. MR 18] Izostavljanje \textlatin{(praetermissio): Cogor hoc loco potissimas eius praetermittere\dots\ felicem meritorum ipsius copiam pauca dicendo uitiare.}
\item[2.11. MR 18–19] Pobožnost kao posljednja spomenuta i posebna vrlina \textlatin{(pietas): Illud unum dicam, cunctis, qui eius consuetudinem nouerunt\dots\ omnium scientiarum libris egregie refertam.}
\begin{description}[nolistsep,itemsep=3pt,font=\rmfamily]
\item[2.11.1. MR 18] Pobožnost prema papi: \textlatin{Illud unum dicam, cunctis, qui eius consuetudinem nouerunt\dots\ nunc mortui desiderio adeo moueatur.}
\item[2.11.2. MR 19] Pobožnost pri obnavljanju i opremanju crkava i samostana: \textlatin{Finem dicendi faciam si prius illud summum eius pietatis munus\dots\ omnium scientiarum libris egregie refertam.}
\end{description}
\item[2.12. MR 19–22] Smrt: \textlatin{Sed dei uoluntate nobis tam repente ademptus est\dots\  ad dominum suum confestim euolauit.}
\begin{description}[nolistsep,itemsep=3pt,font=\rmfamily]
\item[2.12.1. MR 20] Neustrašivost i strpljivost u bolesti \textlatin{(patientia): Cuius miserationis dilectionisque certa indicia\dots\ dolores mira patientia pertulit.}
\item[2.12.2. MR 20] Ispovijest i pokajanje: \textlatin{Delicta, quae uel aetatis uel fortunae\dots\ diuinam uoluntatem accinctus praestolabatur.}
\item[2.12.3. MR 20–21] Govor pred smrt \textlatin{(novissima verba): Iamque uicinus morti\dots\ uel meo exemplo discite.}
\item[2.12.4. MR 22] Opis smrtnog časa: \textlatin{His atque aliis huiusmodi plerisque summa cum religione\dots\ ad dominum suum confestim euolauit.}
\end{description}
\end{description}
\item[3. MR 23 Završetak (peroratio)] pohvala i utjeha: \textlatin{O felix atque iterum felix\dots\ sit nomen domini benedictum. Amen.}
\end{description}

Ključ za razumijevanje strukture daje sam govornik. Već u sredini prve rečenice (MR 1) kao svrhu govora sugerira veličanje dostignuća i vrlina pokojnog prijatelja.

\begin{quote}
\begin{latin}
quo extremum amici munus rebus ab eo bene gestis uirtutumque ipsius copia ac splendore amplissimis laudibus exornarent
\end{latin}

posljednju počast prijatelju uveličati opsežnom pohvalom njegovih dostignuća te obilja i sjaja njegovih vrlina
\end{quote}

U posljednjim rečenicama MR 3, pak, govornik najavljuje kojih će točno šest duhovnih vrlina preminulog kardinala biti predmet hvale.

\begin{quote}
\begin{latin}
habeo tamen et alias immortales ac propemodum diuinas animi ipsius laudes (ut fortunae corporisque quaelibet ingentia bona tamquam aliena relinquam): pietatem, magnitudinem animi, munificentiam, prudentiam, modestiam, atque iustitiam; quae quales in eo fuerint, breuiter explicare conabor.
\end{latin}

ipak znam i druge besmrtne i gotovo božanske odlike njegova duha (da i ne spominjem dobra zadobivena srećom ili tjelesnim sposobnostima; ma kako ona bila silna, ipak su u stanovitoj mjeri tuđa): pobožnost, veličanstvenost duha, darežljivost, razboritost, skromnost i pravednost.  Kakve su one bile u pokojniku, pokušat ću kratko izložiti.
\end{quote}

Kao što se vidi iz pregleda strukture, upravo su vrline Riariju omogućile da uspješno izvršava dužnosti kardinala.

Pritom Nikola pripovjedni dio govora organizira po još jednom načelu: kronološki, od Riarijeva porijekla (MR 3) do smrtnog časa (MR 22). Govor tako ima dvije usporedne okosnice, životopis i vrline. Njihovo vješto isprepletanje vjerojatno je bio jedan od razloga uspjeha djela.

\section{Žanrovska pripadnost}

\section{Stil}

\section{Predaja teksta}

Nikolin govor za Pietra Riarija imao je sedam tiskanih izdanja i šest prijepisa.  Zbog teme djela i poznatih podataka o aktivnosti tiskara, može se pretpostaviti da je \textit{terminus post quem non} tiskanih izdanja, čak i tamo gdje nije naznačen datum, smrt pape Siksta IV, 12. kolovoza 1484.\footnote{Opširniji izvještaj o tiskarima vidi u Neven Jovanović, »Nadgrobni govor Nikole Modruškog za Pietra Riarija« \textit{Colloquia Maruliana} XXVII (2018), str.~123–141.}

Prikazat ćemo najprije tiskana izdanja, potom rukopise.

\subsection{Tiskana izdanja}

U sljedećem popisu navodimo izdanja kronološkim redom (prema datumima potvrđenima u kolofonu ili pretpostavljenima; uglate zagrade označavaju da je godina objavljivanja pretpostavljena), s mjestima i tiskarima. Izdanja označavamo siglom korištenom u kolaciji i kritičkom aparatu. Oznaka iza GW upućuje na zapis u računalnoj bazi podataka \textit{Gesamtkatalog der Wiegendrucke}.\footnote{Staatsbibliothek zu Berlin, Gesamtkatalog der Wiegendrucke / Inkunabelsammlung, www.gesamtkatalogderwiegendrucke.de/. Pristupljeno 29. studenoga 2019.} 

U kronologiji tiskanja skloni smo razlikovati dvije faze; prva bi bila neposredno povezana s Riarijevom smrću, dok drugu smještamo u godine oko 1482, kada se pojavljuje jedino datirano izdanje, ono padovansko Cerdonisovo. Važno je primijetiti da je tada Nikola Modruški već pokojan (umro je prije 29. svibnja 1480) te u ovoj fazi nije mogao imati udjela.

\begin{description}[nolistsep,itemsep=3pt,font=\rmfamily]
\item[V\hphantom{e}] GW M26710, Rim [1474], In domo Antonii et Raphaelis de Vulterris (tiskara djelovala 1472–1474)
\item[V1] GW M26711, Rim [1474], In domo Antonii et Raphaelis de Vulterris\footnote{Mladen Bošnjak ustanovio je 1976, uspoređujući prijelom kolofona i posljednjeg lista primjerka koji se čuva u Trevisu s onim opisanim u katalogu British Library, da postoje dva različita izdanja: »Primjerak iz Trevisa također je proizašao iz oficine Antonii et Raphaelis Vulterris i ono je stvarno drugo izdanje, jer je očito, da tako skladan prelom djela mogao je nastati tek nakon već otisnutog izdanja, a koje je poslužilo kao predložak«. Mladen Bošnjak, »Dvije značajne hrvatske knjižice«, \textit{Hrvatska revija: Jubilarni zbornik 1951–1975}, Knjižnica Hrvatske revije, Barcelona, 1976, 590–598 i Mladen Bošnjak, »Zwei unbekannte Inkunabelausgaben«, \textit{Beiträge zur Inkunabelkunde} 3 (1983), 8, 91–93. Bošnjakova je identifikacija prihvaćena u \textit{Gesamtkatalog der Wiegendrucke}.}
\item[Ge] GW M26707, Rim [1474], Johannes Gensberg (djelovao 1473–1474)
\item[R\hphantom{o}] GW M26712, Rostock [1476], Fratres Domus Horti Viridis ad S.~Michaelem (tiskara djelovala 1476–1531)
\item[C\hphantom{d}] GW M26706, Padova 30.~kolovoza 1482, Matthaeus Cerdonis (djelovao 1482-1487)
\item[P\hphantom{d}] GW M26709, Rim [1482], Stephanus Plannck (djelovao 1479-1501)
\item[Gd] GW M26708, Rim [1482], Bartholomaeus Guldinbeck (djelovao 1475–1488)
\end{description}

Većini tiskara zajednička je orijentacija na mala, ponajviše prigodna izdanja; u taj se profil Nikolin govor savršeno uklapa. Znatan dio angažiranih poduzeća bio je kratka trajanja; tiskare de Vulterris, Gensbergova i Cerdonisova djelovale su pet godina ili manje. Napokon, Antonio, jedan od braće Maffei, potvrđeno je bio povezan s moćnom obitelji Riario, baš kao i Nikola Modruški – možda se, stoga, prvo izdanje govora za Riarija pojavilo upravo \textit{in domo de Vulterris}.\footnote{Braća Maffei radila su na papinskom dvoru kao pisari; bili su aktivni i kao sakupljači rukopisa i kao izdavači. Antonio Maffei sudjelovao je u takozvanoj uroti Pazzijevih. Na vezu tiskara i urotnika upozorila je Farenga, n. dj, 214, bilj. 95.}

\subsection{Rukopisi}

Svih šest danas poznatih rukopisnih prijepisa potječe iz posljednje četvrtine XV.~stoljeća. Četiri se čuvaju u Italiji – u Vatikanu, Rimu, Veneciji i Palermu – dok je po jedan u Münchenu, odnosno u Olomoucu. Svi su prijepisi humanistički zbornici (u njima prevladavaju prozni sastavci). Svi su pisani humanistikom osim onog u Olomoucu, čije je pismo bastarda (iznimno raširena u Češkoj tijekom XIV. i XV. st). Nijedan prijepis nema bitnih naknadnih ispravaka ili marginalnih bilježaka. Također, prema onome što dosad znamo o ruci Nikole Modruškog,\footnote{Luka Špoljarić, »Ex libris Nicolai episcopi Modrussiensis: knjižnica Nikole Modruškoga«, \textit{Colloquia Maruliana} XXI, str.~25–63.} nijedan od prijepisa nije autograf i ne nosi tragove autorskih intervencija.

Donosimo osnovne podatke o prijepisima, uz sigle kojima će kodeksi biti citirani u kritičkom aparatu.

\begin{description}[nolistsep,itemsep=3pt,font=\rmfamily]
\item[ve] Venecija, Biblioteca Marciana; Marc. Lat. cl. XIV, 180 (4667), XV. st, ff.~9r–19v.
\item[va] Vatikan, Bibliotheca apostolica Vaticana; Vat. lat. 8750; kraj XV. – početak XVI. st; ff.~205r–212v. Nekoliko folija ispred govora Modruškog (ff.~162r–172v) u ovom je kodeksu prepisan i nadgrobni govor Niccolòa Perottija (1429–1480) za kardinala Riarija.
\item[co] Rim, Accademia Nazionale dei Lincei, Biblioteca dell'Accademia dei Lincei e Corsiniana, fondo principale, cart. misc., XV. st, Corsin. 583 (45 C 18), ff. 117r–123r. Prepisivač je Tommaso Baldinotti (Pistoia, 1451–1511).\footnote{Identificiran u Armando Petrucci, »Alcuni codici Corsiniani di mano di Tommaso e Antonio Baldinotti«, \textit{Rendiconti dell' Accademia nazionale dei Lincei, classe di scienze morali, storiche e filologiche}, serie 8, XI (1956), str.~252–263.} Prije Nikolina govora (na ff.~113v–116v) prijepis je govora pred Sikstom IV. Nalda Naldija (1436 – nakon 1513), a nakon Nikolina slijedi govor Donata Accaiuolija (1429–1478) pred istim papom (ff.~123v–125v).
\item[pa] Palermo, Biblioteca centrale della Regione siciliana, I.B.6, oko 1474–1480, ff.~32r–54r. Datacija u katalogu određena je upravo prema Nikolinu djelu.
\item[m] München, Bayerische Staatsbibliothek, CLM 461, rukom Hartmanna Schedela (1440–1514), XV. st; ff.~129r–138r. Kodeks, u kojem najmlađi tekst potječe iz 1492, sadrži još govora pred Sikstom IV.\ i govora povodom smrti drugih kardinala; to su djela Bernarda Giustiniana (1408–1489), Lászla Vetésija (r.~oko 1460), Lodovica da Imola (djelovao 1462–1479) u smrt kardinala Pedra Ferriza (u.~1478), anonimni govor u smrt kardinala Spoleta, te govor Gianantonija de San Giorgio (1439–1509) u smrt Ferrija de Clugny, kardinala i biskupa Tournaija (u.~1483).
\item[o] Olomouc, Vědecká knihovna, \textit{Textus uarii; Historia Bohemica,} sign. M I 159, oko 1476–1480, pisan u Litoměřicama;\footnote{Josef Truhlář, \textit{Počátky humanismu v Cechách}, V Praze: Nákladem České akademie císaře Františka Josefa pro vědy, slovesnost a umění, str.~507.} ff.~170r–174v.
\end{description}

\subsection{Kolacija svjedoka i stemma codicum}

\section{Načela ovog izdanja}








\end{document}
\section*{1.\thinspace LIBRI IMPRESSI}


\section*{2.\thinspace CODICES}

\bigskip

\begin{description}[noitemsep,itemsep=3pt,labelsep=5pt,font=\rmfamily] 
\item[ve] -- Codex Venetus, \emph{Marc.~Lat.\ cl.~XIV, 180 (4667),} saec.~XV, ff.~9r–19v.
\item[va] -- Codex Vaticanus: Bibliotheca apostolica Vaticana \emph{Vat.~lat.\ 8750;} saec.\ XV exeunte – XVI ineunte; ff.~205r–212v; e \textit{Ge} descriptus esse uidetur.
\item[pa] -- Codex Panormitanus: Panormi, Biblioteca centrale della Regione siciliana, \emph{I.B.6,} c.~1474–1480; ff.~32r–54r.
\item[co] -- Codex Corsinianus: cart.~misc., saec.~XV; Romae, Accademia Nazionale dei Lincei, Biblioteca dell'Accademia dei Lincei e Corsiniana, fondo principale, \emph{Corsin. 583 (45 C 18),} ff.~117r–123r.
\item[m\hphantom{i}] -- Codex Monacensis: Bibliotheca Bauarica \emph{CLM 461}, ab Hartmanno Schedelio scriptus, saec.~XV; ff.~129r–138r; descriptio \textit{P} uidetur esse.
\item[o\hphantom{o}] -- Codex Olomoucensis, sub fine saec. XV; Olomoucii, Vědecká knihovna; Textus uarii; Historia Bohemica \emph{sign. M I 159}, ff.~170r–174v; e \textit{Ge} descriptus esse uidetur.
\end{description}


\clearpage
\thispagestyle{empty}
\hfill
\clearpage
